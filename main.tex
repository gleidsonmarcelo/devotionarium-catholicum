% chktex-file 8
% chktex-file 19
% chktex-file 38
\documentclass{book}
\usepackage{import}
\import{/home/vatican/devotionarium-catholicum/modules/}{def-packages}
\import{/home/vatican/devotionarium-catholicum/modules/}{def-macros}
\import{/home/vatican/devotionarium-catholicum/modules/}{def-pages}
\geometry{a5paper,hdivide={1.5cm,*,1.5cm},vdivide={1.5cm,*,1.5cm}}
\begin{document}
\begin{center}
    \textcolor{VioletRed2}{\huge{Devocionário Católico}}
\end{center}
\newpage
\begin{center}
    \textbf{Orações da Manhã - Laudes}
\end{center}
\begin{center}
    Sinal da Cruz
\end{center}
\begin{flushleft}
    Pelo Sinal, \grecrossRed{} da Santa Cruz, livrai-nos Deus, \grecrossRed{} Nosso Senhor, dos nossos \grecrossRed{} inimigos. Em nome do Pai, \grecrossRed{} e do Filho, e do Espírito Santo. Amém.
\end{flushleft}
\begin{center}
    Oferecimento do dia
\end{center}
\begin{flushleft}
    Senhor Deus, Rei do céu e da terra, dirige, santifica, conduz e governa, neste dia, nossos corações e nossos corpos, nossos sentimentos, palavras e ações, a fim de que, submissos à tua lei e agindo conforme os teus preceitos, mereçamos, por teu auxílio, ser salvos e livres nesta vida e na eternidade, ó Salvador do mundo, que vives e reinas pelos séculos dos séculos. Amém.
\end{flushleft}
\begin{center}
    Oferecimento de obras
\end{center}
\begin{flushleft}
    Nós Vos rogamos, Senhor, que prepareis as nossas ações com a vossa inspiração, e as acompanheis com a vossa ajuda, a fim de que todos os nossos trabalhos e orações em Vós comecem sempre e convosco acabem. Por Cristo, Senhor Nosso. Amém.
\end{flushleft}
\begin{center}
    À Santíssima Virgem
\end{center}
\begin{flushleft}
    Ó Senhora minha, ó minha Mãe! Eu me ofereço todo a Vós, e, em prova da minha devoção para convosco, Vos consagro neste dia meus olhos, meus ouvidos, minha boca, meu coração e inteiramente todo o meu ser. E como assim sou vosso, ó boa Mãe, guardai-me e defendei-me como coisa e propriedade vossa. Amém.
\end{flushleft}
\begin{center}
    Ao Anjo da Guarda
\end{center}
\begin{flushleft}
    Santo Anjo do Senhor, meu zeloso guardador, se a ti me confiou a piedade divina, sempre me rege e guarda, governa e ilumina. Amém.
\end{flushleft}
\begin{center}
    Glória
\end{center}
\begin{flushleft}
    Glória ao Pai, e ao Filho e ao Espírito Santo. Assim como era no princípio, agora e sempre, por todos os séculos dos séculos. Amém.
\end{flushleft}
\newpage
\begin{center}
    \textbf{Orações do Meio-dia}
\end{center}
\begin{center}
    O Anjo do Senhor (Angelus) \\ \textcolor{VioletRed2}{\scriptsize{(Durante o ano)}}
\end{center}
\begin{flushleft}
    \VbarRed{} O Anjo do Senhor anunciou a Maria. \\
    \RbarRed{} Faça-se em mim segundo a vossa palavra. \\
    \VbarRed{} Ave, Maria, cheia de graça, o Senhor é convosco, bendita sois Vós entre as mulheres e bendito é o fruto do vosso ventre, Jesus. \\
    \RbarRed{} Santa Maria, Mãe de Deus, rogai por nós, pecadores, agora e na hora da nossa morte. Amém.
    \vspace{.2cm} \\
    \VbarRed{} Eis aqui a escrava do Senhor. \\
    \RbarRed{} E Ela concebeu do Espírito Santo. \\
    \VbarRed{} Ave, Maria, cheia de graça, o Senhor é convosco, bendita sois Vós entre as mulheres e bendito é o fruto do vosso ventre, Jesus. \\
    \RbarRed{} Santa Maria, Mãe de Deus, rogai por nós, pecadores, agora e na hora da nossa morte. Amém.
    \vspace{.2cm} \\
    \VbarRed{} E o Verbo de Deus se fez carne. \\
    \RbarRed{} E habitou entre nós. \\
    \VbarRed{} Ave, Maria, cheia de graça, o Senhor é convosco, bendita sois Vós entre as mulheres e bendito é o fruto do vosso ventre, Jesus. \\
    \RbarRed{} Santa Maria, Mãe de Deus, rogai por nós, pecadores, agora e na hora da nossa morte. Amém.
\end{flushleft}
\begin{center}
    Glória \\ \textcolor{VioletRed2}{\scriptsize{(3 Vezes)}}
\end{center}
\begin{flushleft}
    Glória ao Pai, e ao Filho e ao Espírito Santo. Assim como era no princípio, agora e sempre, por todos os séculos dos séculos. Amém.
\end{flushleft}
\newpage
\begin{center}
    Rainha do Céu (Regina Cæli) \\ \textcolor{VioletRed2}{\scriptsize{(Tempo Pascal)}}
\end{center}
\begin{flushleft}
    \VbarRed{} Rainha do Céu, alegrai-Vos, aleluia. \\
    \RbarRed{} Porque quem merecestes trazer em vosso seio, aleluia. \\
    \VbarRed{} Ressuscitou como disse, aleluia. \\
    \RbarRed{} Rogai a Deus por nós, aleluia.
    \vspace{.2cm} \\
    \VbarRed{} Exultai e alegrai-Vos, ó Virgem Maria, aleluia. \\
    \RbarRed{} Porque o Senhor ressuscitou verdadeiramente, aleluia.
    \vspace{.2cm} \\
    \textbf{\textit{Oremos:}} Ó Deus, que Vos dignastes alegrar o mundo com a Ressurreição do vosso Filho Jesus Cristo, Senhor Nosso, concedei-nos, Vos suplicamos, que por sua Mãe, a Virgem Maria, alcancemos as alegrias da vida eterna. Pelo mesmo Jesus Cristo, Senhor Nosso. Amém.
\end{flushleft}
\begin{center}
    Glória \\ \textcolor{VioletRed2}{\scriptsize{(3 Vezes)}}
\end{center}
\begin{flushleft}
    Glória ao Pai, e ao Filho e ao Espírito Santo. Assim como era no princípio, agora e sempre, por todos os séculos dos séculos. Amém.
\end{flushleft}
\newpage
\begin{center}
    \textbf{Orações da Noite - Completas}
\end{center}
\begin{center}
    Sinal da Cruz
\end{center}
\begin{flushleft}
    Pelo Sinal, \grecrossRed{} da Santa Cruz, livrai-nos Deus, \grecrossRed{} Nosso Senhor, dos nossos \grecrossRed{} inimigos. Em nome do Pai, \grecrossRed{} e do Filho, e do Espírito Santo. Amém.
\end{flushleft}
\begin{center}
    Oração da noite
\end{center}
\begin{flushleft}
    Ó Deus eterno, eu Vos adoro e dou graças por todos os vossos benefícios: porque me criaste e remiste por Jesus Cristo e me fizeste cristão e me esperastes, estando eu em pecado, e tantas vezes me tendes perdoado minhas culpas.
\end{flushleft}
\begin{center}
    Ato de Presença de Deus
\end{center}
\begin{flushleft}
    Meu Deus, dai-me luz para conhecer os pecados que hoje cometi, as suas causas e os meios de os evitar.
\end{flushleft}
\begin{center}
    Exame de Consciência
\end{center}
\begin{flushleft}
    Deveres para com Deus: \\ Comecei o dia oferecendo a Deus todos os meus pensamentos, palavras e ações? Fiz algumas outras orações durante o dia: agradecendo, pedindo, oferecendo a Deus o trabalho bem-feito, aceitando com fé os sofrimentos e contrariedades etc.? Procurei viver a fé e cumprir com os preceitos da Igreja? Caí em alguma prática supersticiosa? Fiz alguns pequenos sacrifícios ao comer, ao beber, nas conversas, na guarda da vista pela rua?
    \vspace{.2cm} \\
    Deveres para com o próximo: \\ Tratei os outros com compreensão e paciência? Manifestei-lhes aversão, desprezo ou irritação? Tive inveja? Falei mal da vida alheia, divulgando defeitos ou pecados dos outros? Fui egoísta, pensando só nas atenções que os outros deveriam dar-me, e esquecendo-me de ser prestativo, generoso e dedicado? Admiti sentimentos de ódio, rancor ou vingança? Magoei alguém com brincadeiras e comentários humilhantes? Procurei prestar pequenos serviços aos demais? Fiz o possível por auxiliar os que precisavam de ajuda material ou espiritual, sobretudo no trabalho e em casa? Rezei pelos outros e procurei aproximar algum amigo de Deus? \\
    \newpage
    Deveres para comigo: \\ Esforcei-me por melhorar hoje em alguma virtude, especialmente naquelas em que tenho mais dificuldades? Procurei cumprir com perfeição os meus deveres familiares e profissionais, lutando contra a preguiça, o desleixo, a desordem, o adiamento? Evitei pensamentos, palavras ou atos de orgulho, vaidade, preguiça, sensualidade ou avareza? Deixei-me arrastar pela curiosidade sensual e por desejos impuros? Fui insincero?
    \vspace{.2cm} \\
    Convém lembrar sempre que todos os dias temos de: \\ Glorificar a Deus. Imitar a Jesus Cristo. Invocar a Virgem Santíssima. Implorar a todos os Santos. Salvar a alma. Mortificar o corpo. Adquirir virtudes. Examinar a consciência. Expiar pecados. Evitar o inferno. Ganhar o Paraíso. Preparar a eternidade. Aproveitar o tempo. Edificar o próximo. Preterir o mundo. Combater os demônios. Dominar as paixões. Suportar a morte. Esperar o juízo.
\end{flushleft}
\begin{center}
    Ato de Fé
\end{center}
\begin{flushleft}
    Eu creio firmemente que há um só Deus em três Pessoas realmente distintas, Pai, Filho e Espírito Santo; que dá o Céu aos bons e o inferno aos maus para sempre. Creio que o Filho de Deus se fez homem e morreu na Cruz para nos salvar, e que ao terceiro dia ressuscitou. Creio tudo o mais que crê e ensina a Santa Igreja Católica, Apostólica, Romana, porque Deus, verdade infalível, lho revelou. E nesta crença quero viver e morrer.
\end{flushleft}
\begin{center}
    Ato de Esperança
\end{center}
\begin{flushleft}
    Eu espero, meu Deus, com firme confiança, que pelos merecimentos do meu Senhor Jesus Cristo me dareis a salvação eterna e as graças necessárias para consegui-la, porque Vós, sumamente bom e poderoso, o haveis prometido a quem observar fielmente os vossos mandamentos, como eu me proponho fazer com o vosso auxílio.
\end{flushleft}
\begin{center}
    Ato de Caridade
\end{center}
\begin{flushleft}
    Eu Vos amo, meu Deus, de todo o meu coração e sobre todas as coisas, porque sois infinitamente bom e amável, e antes quero perder tudo do que Vos ofender. Por amor de Vós amo o meu próximo como a mim mesmo.
\end{flushleft}
\newpage
\begin{center}
    Ato de Contrição
\end{center}
\begin{flushleft}
    Eu pecador, me confesso a Deus todo-poderoso, à Bem-aventurada sempre Virgem Maria, ao Bem-aventurado São Miguel Arcanjo, ao Bem-aventurado São João Batista, aos Santos Apóstolos São Pedro e São Paulo, a todos os Santos, e a vós irmãos, porque pequei muitas vezes por pensamentos, palavras e obras, por minha culpa, por minha culpa, minha máxima culpa. Portanto peço, à Bem-aventurada sempre Virgem Maria, ao Bem-aventurado São Miguel Arcanjo, ao Bem-aventurado São João Batista, aos Santos Apóstolos São Pedro e São Paulo, a todos os Santos, e a vós, irmãos, que roqueis por mim a Deus Nosso Senhor. Amém.
\end{flushleft}
\begin{center}
    Oração Particular \\ \textcolor{VioletRed2}{\scriptsize{(Fazer oração particular)}}
\end{center}
\begin{flushleft}
    Eu creio firmemente que há um só Deus em três Pessoas realmente distintas, Pai, Filho e Espírito Santo; que dá o Céu aos bons e o inferno aos maus para sempre. Creio que o Filho de Deus se fez homem e morreu na Cruz para nos salvar, e que ao terceiro dia ressuscitou. Creio tudo o mais que crê e ensina a Santa Igreja Católica, Apostólica, Romana, porque Deus, verdade infalível, lho revelou. E nesta crença quero viver e morrer.
\end{flushleft}
\begin{center}
    Jaculatória
\end{center}
\begin{flushleft}
    Dignai-Vos, Senhor, retribuir com a vida eterna a todos os que nos fazem bem por amor do vosso nome.
\end{flushleft}
\begin{center}
    Glória
\end{center}
\begin{flushleft}
    Glória ao Pai, e ao Filho e ao Espírito Santo. Assim como era no princípio, agora e sempre, por todos os séculos dos séculos. Amém.
\end{flushleft}
\newpage
\begin{center}
    \textbf{Meditação}
\end{center}
\begin{center}
    Vinde, Espírito Santo
\end{center}
\begin{flushleft}
    Vinde, Espírito Santo, enchei os corações dos vossos fiéis e acendei neles o fogo do vosso amor. \\
    \VbarRed{} Rainha do Céu, alegrai-Vos, aleluia. \\
    \RbarRed{} Porque quem merecestes trazer em vosso seio, aleluia.
    \vspace{.2cm} \\
    \textbf{\textit{Oremos:}} Ó Deus, que instruístes os corações dos vossos fiéis com a luz do Espírito Santo, concedei-nos amar, no mesmo Espírito, o que é reto, e gozar sempre a sua consolação. Por Cristo, Senhor Nosso. Amém.
    \vspace{.2cm} \\
    \textit{Antes:} \\ Meu Senhor e Meu Deus, creio firmemente que estás aqui, que me vês, que me ouves. Adoro-Te com profunda reverência. Peço-Te perdão dos meus pecados e graça para fazer com fruto este tempo de oração. \\ Minha Mãe Imaculada, São José, meu Pai e Senhor, meu Anjo da Guarda, intercedei por mim.
    \vspace{.2cm} \\
    \textit{Depois:} \\ Dou-Te graças, Meu Deus, pelos bons propósitos, afetos e inspirações que me comunicaste nesta meditação; peço-Te ajuda para os pôs em prática. \\ Minha Mãe Imaculada, São José, meu Pai e Senhor, meu Anjo da Guarda, intercedei por mim.
\end{flushleft}
\newpage
\begin{center}
    \textbf{Bênção dos Alimentos}
\end{center}
\begin{center}
    Bênção antes das refeições
\end{center}
\begin{flushleft}
    Abençoai-nos, Senhor, a nós e a estes dons que da vossa liberalidade recebemos. Por Cristo, Senhor Nosso. Amém.
    \vspace{.2cm} \\
    \textit{Almoço:} \\
    \VbarRed{} Que o Rei da eterna glória nos faça participantes da mesa celestial. \\
    \RbarRed{} Amém.
    \vspace{.2cm} \\
    \textit{Jantar:} \\
    \VbarRed{} Que o Rei da eterna glória nos conduza à Ceia da vida eterna. \\
    \RbarRed{} Amém.
    \vspace{.2cm} \\
    Em nome do Pai, \grecrossRed{} e do Filho, e do Espírito Santo. Amém.
\end{flushleft}
\begin{center}
    Benção depois das refeições
\end{center}
\begin{flushleft}
    Nós Vos damos graças, Deus onipotente, por todos vossos benefícios, Vós que viveis e reinais por todos os séculos dos séculos. Amém.
    \vspace{.2cm} \\
    \VbarRed{} Que Deus nos dê a sua paz. \\
    \RbarRed{} E a vida eterna. Amém.
    \vspace{.2cm} \\
    Em nome do Pai, \grecrossRed{} e do Filho, e do Espírito Santo. Amém.
\end{flushleft}
\newpage
\begin{center}
    \textbf{Santo Rosário}
\end{center}
\begin{center}
    Sinal da Cruz
\end{center}
\begin{flushleft}
    Pelo Sinal, \grecrossRed{} da Santa Cruz, livrai-nos Deus, \grecrossRed{} Nosso Senhor, dos nossos \grecrossRed{} inimigos. Em nome do Pai, \grecrossRed{} e do Filho, e do Espírito Santo. Amém.
\end{flushleft}
\begin{center}
    Jaculatórias
\end{center}
\begin{flushleft}
    \VbarRed{} Abri os meus lábios, ó Senhor. \\
    \RbarRed{} E minha boca anunciará vosso louvor.
    \vspace{.2cm} \\
    \VbarRed{} Vinde, ó Deus em meu auxílio. \\
    \RbarRed{} Socorrei-me sem demora.
\end{flushleft}
\begin{center}
    Vinde, Espírito Santo
\end{center}
\begin{flushleft}
    Vinde, Espírito Santo, enchei os corações dos vossos fiéis e acendei neles o fogo do vosso amor. \\
    \VbarRed{} Rainha do Céu, alegrai-Vos, aleluia. \\
    \RbarRed{} Porque quem merecestes trazer em vosso seio, aleluia.
    \vspace{.2cm} \\
    \textbf{\textit{Oremos:}} Ó Deus, que instruístes os corações dos vossos fiéis com a luz do Espírito Santo, concedei-nos amar, no mesmo Espírito, o que é reto, e gozar sempre a sua consolação. Por Cristo, Senhor Nosso. Amém. \\
\end{flushleft}
\begin{center}
    Oferecimento
\end{center}
\begin{flushleft}
    Divino Jesus, eu vos ofereço este rosário que vou rezar, contemplando os mistérios da nossa redenção. Concedei-me, pela intercessão de Maria, Vossa Mãe Santíssima, a quem me dirijo, as virtudes que me são necessárias para rezá-lo bem e a graça de ganhar as indulgências anexas a esta santa devoção.
\end{flushleft}
\begin{center}
    Jaculatória
\end{center}
\begin{flushleft}
    \VbarRed{} Dignai-vos, ó Virgem Sagrada, consentir em ser por mim louvada. \\
    \RbarRed{} Dai-me força contra os vossos inimigos.
\end{flushleft}
\begin{center}
    Credo
\end{center}
\begin{flushleft}
    Creio em Deus Pai Todo-poderoso, criador do céu e da terra; e em Jesus Cristo, seu único Filho, Nosso Senhor, que foi concebido pelo poder do Espírito Santo; nasceu da Virgem Maria, padeceu sob Pôncio Pilatos, foi crucificado, morto e sepultado; desceu à mansão dos mortos; ressuscitou ao terceiro dia; subiu aos céus, está sentado à direita de Deus Pai Todo-poderoso, donde há de vir a julgar os vivos e os mortos; creio no Espírito Santo, na santa Igreja Católica, na comunhão dos santos, na remissão dos pecados, na ressurreição da carne, na vida eterna. Amém.
\end{flushleft}
\begin{center}
    Pai Nosso
\end{center}
\begin{flushleft}
    Pai nosso que estais nos céus, santificado seja o vosso nome; venha a nós o vosso reino, seja feita a vossa vontade assim na terra como no céu. O pão nosso de cada dia nos daí hoje; perdoai-nos as nossas ofensas, assim como nós perdoamos a quem nos tem ofendido, e não nos deixeis cair em tentação, mas livrai-nos do mal. Amém.
\end{flushleft}
\begin{center}
    Ave Maria \\ \textcolor{VioletRed2}{\scriptsize{(3 Vezes)}}
\end{center}
\begin{flushleft}
    Ave, Maria, cheia de graça, o Senhor é convosco, bendita sois Vós entre as mulheres e bendito é o fruto do vosso ventre, Jesus. Santa Maria, Mãe de Deus, rogai por nós, pecadores, agora e na hora da nossa morte. Amém.
\end{flushleft}
\begin{center}
    Glória
\end{center}
\begin{flushleft}
    Glória ao Pai, e ao Filho e ao Espírito Santo. Assim como era no princípio, agora e sempre, por todos os séculos dos séculos. Amém.
    \vspace{.2cm} \\
    \VbarRed{} Ó meu Jesus, perdoai-nos, livrai-nos do fogo do inferno. \\
    \RbarRed{} Levai as almas todas para o Céu e socorrei principalmente as que mais precisarem.
    \vspace{.2cm} \\
    \VbarRed{} Ó Maria concebida sem pecado.\\
    \RbarRed{} Rogai por nós, que recorremos a Vós.
\end{flushleft}
\newpage
\begin{center}
    Mistérios Gozosos \\ \textcolor{VioletRed2}{\scriptsize{(Segunda-Feira, Quinta-Feira, Domingos do Advento e Natal)}} \\
    \hfill{} \break{}
    1\textordmasculine{} Mistério \\ Anunciação do Anjo e a Encarnação do Verbo no seio Puríssimo da Virgem Maria
\end{center}
\begin{flushleft}
    Do Evangelho segundo São Lucas - (\textcolor{VioletRed2}{Lc 1,26-38}). \\
    \hfill{} \break{}
    Quando Isabel estava no sexto mês, o anjo Gabriel foi enviado por deus a uma cidade da Galileia, chamada Nazaré, a uma virgem prometida em casamento a um homem de nome José, da casa de Davi. O nome da virgem era Maria. O anjo entrou onde ela estava e disse: ``Alegra-te, cheia de graça! O Senhor está contigo''. Ela perturbou-se com essas palavras e pôs-se a pensar no que significaria a saudação. O anjo, então, disse: ``Não temas, Maria! Encontraste graça junto a Deus. Conceberás e darás à luz um filho, e lhe porás o nome de Jesus. Ele será grande e será chamado Filho do Altíssimo, e o Senhor Deus lhe dará o trono de seu pai Davi. Ele reinará para sempre sobre a casa de Jacó, e o seu reino não terá fim''.
    \vspace{.2cm} \\
    Maria, então, perguntou ao anjo: ``Como acontecerá isso, se não conheço homem algum?'' O anjo respondeu: ``O Espírito Santo descerá sobre ti, e o poder do Altíssimo te cobrirá com sua sombra. Por isso, aquele que vai nascer é santo e será chamado Filho de Deus. Também Isabel, tua parenta, concebeu um filho na sua velhice, já está no sexto mês aquela que era chamada estéril, pois \textit{para Deus nada é impossível}''. Então Maria disse: ``Eis aqui a serva do Senhor! Faça-se em mim segundo a tua palavra''. E o anjo saiu da sua presença. \\
    \hfill{} \break{}
    Pai Nosso, dez Ave Maria, Glória.
\end{flushleft}
\newpage
\begin{center}
    2\textordmasculine{} Mistério \\ Visitação da Virgem Maria a sua prima Santa Isabel e Santificação de São João Batista
\end{center}
\begin{flushleft}
    Do Evangelho segundo São Lucas - (\textcolor{VioletRed2}{Lc 1,39-42}). \\
    \hfill{} \break{}
    Naqueles dias, Maria levantou-se e foi apressadamente à região montanhosa, a uma cidade de Judá. Ela entrou na casa de Zacarias e saudou Isabel. Quando Isabel ouviu a saudação de Maria, a criança saltou de alegria em seu ventre. Isabel ficou repleta de Espírito Santo e, com voz forte, exclamou: ``Bendita és tu entre as mulheres e bendito é o fruto do teu ventre! Como me acontece que a mãe do meu Senhor venha a mim? Logo que ressoou aos meus ouvidos a tua saudação, a criança pulou de alegria no meu ventre. Bem-aventurada aquela que acreditou, porque se cumprirá o que lhe foi dito da parte do Senho''. \\
    \hfill{} \break{}
    Pai Nosso, dez Ave Maria, Glória.
\end{flushleft}
\begin{center}
    3\textordmasculine{} Mistério \\ Nascimento do Menino Jesus em Belém
\end{center}
\begin{flushleft}
    Do Evangelho segundo São Lucas - (\textcolor{VioletRed2}{Lc 2,1-7}). \\
    \hfill{} \break{}
    Naqueles dias foi publicado um decreto do imperador Augusto ordenando o recenseamento do mundo inteiro. Esse primeiro recenseamento aconteceu quando Quirino era governador da Síria. Todos iam registrar-se, cada um em sua própria cidade. Também José - que era da casa e da linhagem de Davi - subiu da Galileia, da cidade de Nazaré, à Judeia, à cidade de Davi, chamada Belém, para registrar-se com Maria, sua esposa, que estava grávida. Quando estavam ali, completaram-se os dias de ela dar à luz. Ela deu à luz o seu filho, o primogênito, envolveu-o em faixas e deitou-o numa manjedoura, porque não havia lugar para eles na hospedaria. \\
    \hfill{} \break{}
    Pai Nosso, dez Ave Maria, Glória.
\end{flushleft}
\newpage
\begin{center}
    4\textordmasculine{} Mistério \\ Apresentação do Menino Jesus no templo e a Purificação da Virgem Maria
\end{center}
\begin{flushleft}
    Do Evangelho segundo São Lucas - (\textcolor{VioletRed2}{Lc 2,21-24}). \\
    \hfill{} \break{}
    Completados os oitos dias para a circuncisão do menino, deram-lhe o nome de Jesus, como fora chamado pelo anjo, antes de ser concebido no ventre de sua mãe.
    \vspace{.2cm} \\
    Quando se completaram os dias da purificação, segundo a lei de Moisés, levaram o menino a Jerusalém para apresentá-lo ao Senhor, conforme está escrito na Lei do Senhor: ``\textit{Todo primogênito masculino será consagrado ao Senhor}''; e para oferecer em sacrifício \textit{um par de rolas ou dois pombinhos}, como está escrito na Lei do Senhor. \\
    \hfill{} \break{}
    Pai Nosso, dez Ave Maria, Glória.
\end{flushleft}
\begin{center}
    5\textordmasculine{} Mistério \\ Perda e o Encontro do Menino Jesus no Templo discutindo com os doutores da Lei
\end{center}
\begin{flushleft}
    Do Evangelho segundo São Lucas - (\textcolor{VioletRed2}{Lc 2,41-50}). \\
    \hfill{} \break{}
    Todos os anos, os pais de Jesus iam a Jerusalém para a festa da Páscoa. Quando ele completou doze anos, subiram para a festa, como de costume. Terminados os dias da festa, no momento de voltarem, Jesus permaneceu em Jerusalém, sem que seus pais percebessem. Pensando que se encontrasse na caravana, fizeram o caminho de um dia e procuravam-no entre os parentes e conhecidos; mas, como não o encontrassem, voltaram a Jerusalém, à procura dele. Depois de três dias o encontraram no templo, sentado entre os mestres ouvindo-os e fazendo-lhes perguntas. Todos os que ouviam o menino ficavam extasiados com sua inteligência e suas repostas. Quando o viram, seus pais ficaram admirados, e sua mãe lhe disse: ``Filho, por que agiste assim conosco? Olha, teu pai e eu andávamos, angustiados, à tua procura!'' Ele respondeu: ``Por que me procuráveis? Não sabeis que eu devo estar naquilo que é de meu Pai?'' Eles, porém, não entenderam o que ele lhes havia dito. \\
    \hfill{} \break{}
    Pai Nosso, dez Ave Maria, Glória.
\end{flushleft}
\newpage
\begin{center}
    Mistérios Dolorosos \\ \textcolor{VioletRed2}{\scriptsize{(Terça-Feira, Sexta-Feira, Domingos da Quaresma e Tríduo Pascal)}} \\
    \hfill{} \break{}
    1\textordmasculine{} Mistério \\ Agonia de Nosso Senhor Jesus Cristo e a oração no Jardim das Oliveiras
\end{center}
\begin{flushleft}
    Do Evangelho segundo São Marcos - (\textcolor{VioletRed2}{Mc 14,32-36}). \\
    \hfill{} \break{}
    Chegaram a um lugar chamado Getsêmani. Jesus disse aos discípulos: ``Sentai-vos aqui, enquanto eu vou orar''. Levou consigo Pedro, Tiago e João, e começou a sentir pavor e angústia. Ele disse-lhes: ``Minha alma está triste até a morte! Ficai aqui e vigiai''! Então foi um pouco mais adiante, caiu por terra e orava para que, se possível, passasse dele aquela hora. E dizia: ``Abá, Pai! Tudo te é possível. Afasta de mim este cálice! Contudo, não seja o que eu quero, mas o que tu queres''. \\
    \hfill{} \break{}
    Pai Nosso, dez Ave Maria, Glória.
\end{flushleft}
\begin{center}
    2\textordmasculine{} Mistério \\ Flagelação de Nosso Senhor Jesus Cristo
\end{center}
\begin{flushleft}
    Do Evangelho segundo São Mateus - (\textcolor{VioletRed2}{Mt 27,22-26}). \\
    \hfill{} \break{}
    Pilatos perguntou: ``Que farei, então, com Jesus, que é chamado o Cristo?'' Todos responderam: ``Seja crucificado!'' Pilatos falou: ``Mas, que mal ele fez?'' Eles, porém, gritaram com mais força: ``Seja crucificado!'' Quando Pilatos viu que nada conseguia e que, ao contrário, aumentava o tumulto, mandou trazer água, lavou as mãos diante da multidão e disse: ``Sou inocente do sangue deste homem. A responsabilidade é vossa!'' O povo todo respondeu: ``Que o sangue dele recaia sobre nós e sobre nossos filhos''. Então Pilatos soltou Barrabás, mandou flagelar Jesus e entregou-o para ser crucificado. \\
    \hfill{} \break{}
    Pai Nosso, dez Ave Maria, Glória.
\end{flushleft}
\newpage
\begin{center}
    3\textordmasculine{} Mistério \\ Nosso Senhor Jesus Cristo é coroado de espinhos
\end{center}
\begin{flushleft}
    Do Evangelho segundo São Mateus - (\textcolor{VioletRed2}{Mt 27,27-31}). \\
    \hfill{} \break{}
    Em seguida, os soldados do governador levaram Jesus ao pretório e reuniram toda a guarnição em volta dele. Tiraram-lhe as vestes e vestiram-no com um manto escarlate. Depois, puseram-lhe na cabeça uma coroa de espinhos que trançaram, e um caniço na mão direita, e ajoelharam-se diante de Jesus, enquanto diziam, zombando: ``Salve, rei dos judeus!'' Cuspiram nele e bateram-lhe na cabeça com o caniço. Depois de zombar dele, tiraram-lhe o manto e o vestiram com suas próprias vestes. \\
    \hfill{} \break{}
    Pai Nosso, dez Ave Maria, Glória.
\end{flushleft}
\begin{center}
    4\textordmasculine{} Mistério \\ Nosso Senhor Jesus Cristo carrega a Cruz nas costas a caminho do Calvário
\end{center}
\begin{flushleft}
    Do Evangelho segundo São João - (\textcolor{VioletRed2}{Jo 19,16-22}). \\
    \hfill{} \break{}
    Pilatos, então, lhes entregou Jesus para ser crucificado.
    \vspace{.2cm} \\
    Então receberam Jesus. E, carregando ele próprio sua cruz, saiu para o lugar chamado Calvário, em hebraico: Gólgota. Lá, eles o crucificaram com outros dois, um de cada lado, e Jesus no meio. Pilatos mandou escrever e afixar na cruz um letreiro; estava escrito assim: ``Jesus Nazareno, o Rei dos Judeus''. Como o lugar onde Jesus fora crucificado, era perto da cidade, muitos judeus leram o letreiro, que estava escrito em hebraico, latim e grego. Os chefes dos sacerdotes dos judeus disseram então a Pilatos: ``Não escrevas: `O Rei dos Judeus' e sim: `Ele disse: Eu sou o Rei dos Judeus'\,''. Pilatos respondeu: ``O que escrevi, escrevi''. \\
    \hfill{} \break{}
    Pai Nosso, dez Ave Maria, Glória.
\end{flushleft}
\newpage
\begin{center}
    5\textordmasculine{} Mistério \\ Crucifixão e Morte de Nosso Senhor Jesus Cristo
\end{center}
\begin{flushleft}
    Do Evangelho segundo São João - (\textcolor{VioletRed2}{Jo 19,23-30}). \\
    \hfill{} \break{}
    Depois de crucificarem Jesus, os soldados pegaram suas vestes e as dividiram em quatro partes, para cada soldado uma parte. Pegaram também a túnica, que era feita sem costura, uma peça única, de alto a baixo, e combinaram: ``Não vamos rasgar a túnica, vamos tirar a sorte para ver de quem será''. Assim cumpriu-se a Escritura:
    \vspace{.2cm} \\
    ``\textit{Repartiram entre si as minhas vestes e tiraram a sorte sobre minha túnica}''. Foi o que os soldados fizeram.
    \vspace{.2cm} \\
    Junto à cruz de Jesus estavam de pé sua mãe e a irmã de sua mãe, Maria de Cléofas, e Maria Madalena. Jesus, ao ver sua mãe e, ao lado dela, o discípulo a quem amava, disse à mãe: ``Mulher, eis o teu filho!'' Depois disse ao discípulo: ``Eis tua mãe!'' A partir daquela hora, o discípulo a acolheu em sua casa.
    \vspace{.2cm} \\
    Em seguida, sabendo Jesus que tudo estava consumado, para que se cumprisse a Escritura, disse: ``Tenho sede''! Havia ali uma vasilha cheia de vinagre. Fixaram uma esponja embebida em vinagre num ramo de hissopo e a levaram à sua boca. Depois que tomou o vinagre, ele disse: ``Está consumado''. E, inclinando a cabeça, entregou o espírito. \\
    \hfill{} \break{}
    Pai Nosso, dez Ave Maria, Glória.
\end{flushleft}
\newpage
\begin{center}
    Mistérios Mistérios Gloriosos \\ \textcolor{VioletRed2}{\scriptsize{(Quarta-Feira, Sábado, Domingos da Páscoa e Tempo Comum)}} \\
    \hfill{} \break{}
    1\textordmasculine{} Mistério \\ Ressurreição de Nosso Senhor Jesus Cristo
\end{center}
\begin{flushleft}
    Do Evangelho segundo São Marcos - (\textcolor{VioletRed2}{Mc 16,9-15}). \\
    \hfill{} \break{}
    Tendo ressuscitado, na madrugada do primeiro dia da semana, Jesus apareceu primeiro a Maria Madalena, de quem tinha expulsado sete demônios. Ela foi anunciar o fato aos que acompanharam Jesus, e que estavam aflitos e choravam. Quando ouviram que ele estava vivo e havia sido visto por ela, não acreditaram. Depois disso, Jesus, de aspecto mudado, apareceu a dois deles, enquanto estavam a caminho do campo. Estes voltaram para anunciá-lo aos outros, os quais tampouco acreditaram.
    \vspace{.2cm} \\
    Por fim, Jesus apareceu aos onze discípulos, enquanto à mesa. Ele os censurou por sua falta de fé e sua dureza de coração, por não terem acreditado naqueles que o tinham visto ressuscitado. Então disse-lhes: ``Ide pelo mundo inteiro e proclamai o Evangelho a toda criatura!
    \vspace{.2cm} \\
    Quem crer e for batizado será salvo, mas quem não crer, será condenado. Eis os sinais que acompanharão aqueles que crerem: expulsarão demônios em meu nome; falarão novas línguas; se pegarem em serpentes e beberem veneno mortal, não lhes fará mal algum; e quando impuserem as mãos sobre os enfermos, estes ficarão curados''. \\
    \hfill{} \break{}
    Pai Nosso, dez Ave Maria, Glória.
\end{flushleft}
\begin{center}
    2\textordmasculine{} Mistério \\ Ascensão de Nosso Senhor Jesus Cristo
\end{center}
\begin{flushleft}
    Do Evangelho segundo São Marcos - (\textcolor{VioletRed2}{Mc 16,19-20}). \\
    \hfill{} \break{}
    Depois de falar com os discípulos, o Senhor Jesus foi elevado ao céu e sentou-se à direita de Deus.
    \vspace{.2cm} \\
    Então, os discípulos foram anunciar por toda parte. O Senhor cooperava, confirmando a palavra pelos sinais que a acompanhavam. \\
    \hfill{} \break{}
    Pai Nosso, dez Ave Maria, Glória.
\end{flushleft}
\newpage
\begin{center}
    3\textordmasculine{} Mistério \\ Descida do Espírito Santo sobre a Virgem Maria e os Apóstolos reunidos no Cenáculo
\end{center}
\begin{flushleft}
    Do Evangelho segundo São João - (\textcolor{VioletRed2}{Jo 14,25-31}). \\
    \hfill{} \break{}
    Eu vos tenho falado dessas coisas estando ainda convosco. Ora, o Paráclito, o Espírito Santo que o Pai enviará em meu nome, ele vos ensinará tudo e vos recordará tudo o que eu vos tenho dito. Deixo-vos a paz, dou-vos a minha paz. Eu não a dou, como a dá o mundo. Não se perturbe, nem se atemorize o vosso coração. Ouvistes o que vos disse: `Eu vou, mas voltarei a vós'. Se me amásseis, ficaríeis alegres porque vou para o Pai, pois o Pai é maior do que eu.
    \vspace{.2cm} \\
    Disse-vos isso agora, antes que aconteça, para que, quando acontecer, creiais. Já não falarei mais convosco, pois vem aí o príncipe deste mundo. Ele nada pode contra mim, mas é preciso que o mundo saiba que eu amo o Pai, e faço como o Pai me mandou. Levantai-vos! Vamo-nos daqui!'' \\
    \hfill{} \break{}
    Pai Nosso, dez Ave Maria, Glória.
\end{flushleft}
\begin{center}
    4\textordmasculine{} Mistério \\ Assunção de Nossa Senhora aos Céus de corpo e alma
\end{center}
\begin{flushleft}
    Do Evangelho segundo São Lucas - (\textcolor{VioletRed2}{Lc 11,27-28}). \\
    \hfill{} \break{}
    Enquanto Jesus assim falava, uma mulher levantou a voz, do meio da multidão, e lhe disse: ``Bem-aventurado o ventre que te gerou e os seios que te amamentaram''. Ele respondeu: ``Bem-aventurados, antes, os que ouvem a Palavra de Deus e a guardam''. \\
    \hfill{} \break{}
    Pai Nosso, dez Ave Maria, Glória.
\end{flushleft}
\newpage
\begin{center}
    5\textordmasculine{} Mistério \\ Coroação de Nossa Senhora como Rainha do Céu e da Terra
\end{center}
\begin{flushleft}
    Do Evangelho segundo São Lucas - (\textcolor{VioletRed2}{Lc 1,46-55}). \\
    \hfill{} \break{}
    ``\textit{A minha alma engrandece o Senhor}, e meu espírito exulta \textit{em Deus, meu Salvador}, porque olhou \textit{para a condição humilde de sua serva.} Todas as gerações, desde agora, me chamarão bem-aventurada, porque o Poderoso fez por mim grandes coisas. Santo é o seu nome, e sua misericórdia se estende, de geração em geração, sobre aqueles que o temem. Ele manifestou poder com o seu braço: dispersou os soberbos nos pensamentos de seu coração. Depôs os poderosos de seus tronos e exaltou os de condição humilde. Encheu de bens os famintos e despediu os ricos sem nada. Amparou Israel, seu servo, lembrando-se da misericórdia, como prometera a nossos pais, a Abraão e à sua descendência, para sempre. \\
    \hfill{} \break{}
    Pai Nosso, dez Ave Maria, Glória.
\end{flushleft}
\begin{center}
    Agradecimento
\end{center}
\begin{flushleft}
    Infinitas graças vos damos, soberana Rainha, pelos benefícios que todos os dias recebemos das vossas mãos liberais. Dignai-vos, agora e sempre, tomar-nos debaixo do vosso poderoso amparo, e para mais vos obrigar, vos saudamos com uma Salve Rainha.
\end{flushleft}
\begin{center}
    Salve Rainha
\end{center}
\begin{flushleft}
    Salve Rainha, Mãe de misericórdia, vida, doçura, esperança nossa, salve! A vós bradamos, os degredados filhos de Eva. A Vós suspiramos, gemendo e chorando neste vale de lágrimas. Eia, pois advogada nossa, esses vossos olhos misericordiosos a nós volvei, e depois deste desterro mostrai-nos Jesus, bendito fruto do vosso ventre. Ó clemente, ó piedosa, ó doce sempre Virgem Maria.
\end{flushleft}
\newpage
\begin{center}
    Ladainha Lauretana
\end{center}
\begin{flushleft}
    \VbarRed{} Senhor, tende piedade de nós. \\
    \RbarRed{} Senhor, tende piedade de nós. \\
    \VbarRed{} Cristo, tende piedade de nós. \\
    \RbarRed{} Cristo, tende piedade de nós. \\
    \VbarRed{} Senhor, tende piedade de nós. \\
    \RbarRed{} Senhor, tende piedade de nós.
    \vspace{.2cm} \\
    \VbarRed{} Cristo, ouvi-nos. \\
    \RbarRed{} Cristo, ouvi-nos. \\
    \VbarRed{} Cristo, atendei-nos. \\
    \RbarRed{} Cristo, atendei-nos.
    \vspace{.2cm} \\
    \VbarRed{} Deus pai do Céu. \\
    \RbarRed{} Tende piedade de nós. \\
    \VbarRed{} Deus Filho Redentor do mundo. \\
    \RbarRed{} Tende piedade de nós. \\
    \VbarRed{} Deus Espírito Santo. \\
    \RbarRed{} Tende piedade de nós. \\
    \VbarRed{} Santíssima Trindade, que sois um só Deus. \\
    \RbarRed{} Tende piedade de nós.
    \vspace{.2cm} \\
    \VbarRed{} Santa Maria. \\
    \RbarRed{} Rogai por nós.
    \vspace{.2cm} \\
    \textcolor{VioletRed2}{\small{(Repete-se a mesma resposta a partir daqui)}}
    \vspace{.2cm} \\
    \VbarRed{} Santa Mãe de Deus, \\
    \VbarRed{} Santa Virgem das virgens, \\
    \VbarRed{} Mãe de Cristo, \\
    \VbarRed{} Mãe da Igreja, \\
    \VbarRed{} Mãe da misericórdia, \\
    \VbarRed{} Mãe da divina graça, \\
    \VbarRed{} Mãe da esperança, \\
    \VbarRed{} Mãe puríssima, \\
    \VbarRed{} Mãe castíssima, \\
    \VbarRed{} Mãe sempre virgem, \\
    \VbarRed{} Mãe imaculada, \\
    \VbarRed{} Mãe digna de amor, \\
    \VbarRed{} Mãe admirável, \\
    \VbarRed{} Mãe do bom conselho, \\
    \VbarRed{} Mãe do Criador, \\
    \VbarRed{} Mãe do Salvador, \\
    \VbarRed{} Virgem prudentíssima, \\
    \VbarRed{} Virgem venerável, \\
    \VbarRed{} Virgem louvável, \\
    \VbarRed{} Virgem poderosa, \\
    \VbarRed{} Virgem clemente, \\
    \VbarRed{} Virgem fiel, \\
    \VbarRed{} Espelho da perfeição, \\
    \VbarRed{} Sede da sabedoria, \\
    \VbarRed{} Fonte da nossa alegria, \\
    \VbarRed{} Vaso espiritual, \\
    \VbarRed{} Tabernáculo da eterna glória, \\
    \VbarRed{} Moradia consagrada a Deus, \\
    \VbarRed{} Rosa mística, \\
    \VbarRed{} Torre de Davi, \\
    \VbarRed{} Torre de marfim, \\
    \VbarRed{} Casa de ouro, \\
    \VbarRed{} Arca da aliança, \\
    \VbarRed{} Porta do Céu, \\
    \VbarRed{} Estrela da manhã, \\
    \VbarRed{} Saúde dos enfermos, \\
    \VbarRed{} Refúgio dos pecadores, \\
    \VbarRed{} Socorro dos migrantes, \\
    \VbarRed{} Consoladora dos aflitos, \\
    \VbarRed{} Auxílio dos cristãos, \\
    \VbarRed{} Rainha dos Anjos, \\
    \VbarRed{} Rainha dos Patriarcas, \\
    \VbarRed{} Rainha dos Profetas, \\
    \VbarRed{} Rainha dos Apóstolos, \\
    \VbarRed{} Rainha dos Mártires, \\
    \VbarRed{} Rainha dos Confessores da fé, \\
    \VbarRed{} Rainha das Virgens, \\
    \VbarRed{} Rainha de todos os Santos, \\
    \VbarRed{} Rainha concebida sem pecado original, \\
    \VbarRed{} Rainha assunta aos Céus, \\
    \VbarRed{} Rainha do Santo Rosário, \\
    \VbarRed{} Rainha da Paz.
    \vspace{.2cm} \\
    \VbarRed{} Cordeiro de Deus, que tirais o pecado do mundo. \\
    \RbarRed{} Perdoai-nos, Senhor. \\
    \VbarRed{} Cordeiro de Deus, que tirais o pecado do mundo. \\
    \RbarRed{} Ouvi-nos, Senhor. \\
    \VbarRed{} Cordeiro de Deus, que tirais o pecado do mundo. \\
    \RbarRed{} Tende piedade de nós.
    \newpage
    À Vossa proteção nos acolhemos, Santa Mãe de Deus; não desprezeis as súplicas que em nossas necessidades Vos dirigimos, mas livrai-nos sempre de todos os perigos, ó Virgem gloriosa e bendita.
    \vspace{.2cm} \\
    \VbarRed{} Rogai por nós, Santa Mãe de Deus. \\
    \RbarRed{} Para que sejamos, dignos das promessas de Cristo.
    \vspace{.2cm} \\
    \textbf{\textit{Oremos:}} Infundi, Senhor, nós Vos pedimos, em nossas almas a vossa graça, para que nós, que conhecemos pela Anunciação do Anjo a Encarnação de Jesus Cristo, vosso filho, cheguemos por sua Paixão e sua Cruz à glória da Ressurreição. Pelo mesmo Jesus Cristo, Senhor Nosso.
    \vspace{.2cm} \\
    Em nome do Pai, \grecrossRed{} e do Filho, e do Espírito Santo. Amém.
\end{flushleft}
\end{document}
