% chktex-file 8
% chktex-file 19
% chktex-file 36
% chktex-file 38
\documentclass{book}
\usepackage{import}
\import{/home/vatican/devotionarium-catholicum/modules/}{def-packages}
\import{/home/vatican/devotionarium-catholicum/modules/}{def-macros}
\import{/home/vatican/devotionarium-catholicum/modules/}{def-pages}
\geometry{a5paper,hdivide={1.5cm,*,1.5cm},vdivide={1.5cm,*,1.5cm}}
\begin{document}
\begin{center}
    \textcolor{VioletRed2}{\huge{Devocionário Católico}}
\end{center}
\newpage
\begin{center}
    \textbf{Orações da Manhã - Laudes}
\end{center}
\begin{center}
    Sinal da Cruz
\end{center}
\begin{flushleft}
    Pelo Sinal, \grecrossRed{} da Santa Cruz, livrai-nos Deus, \grecrossRed{} Nosso Senhor, dos nossos \grecrossRed{} inimigos. Em nome do Pai, \grecrossRed{} e do Filho, e do Espírito Santo. Amém.
\end{flushleft}
\begin{center}
    Oferecimento do dia
\end{center}
\begin{flushleft}
    Senhor Deus, Rei do céu e da terra, dirige, santifica, conduz e governa, neste dia, nossos corações e nossos corpos, nossos sentimentos, palavras e ações, a fim de que, submissos à tua lei e agindo conforme os teus preceitos, mereçamos, por teu auxílio, ser salvos e livres nesta vida e na eternidade, ó Salvador do mundo, que vives e reinas pelos séculos dos séculos. Amém.
\end{flushleft}
\begin{center}
    Oferecimento de obras
\end{center}
\begin{flushleft}
    Nós Vos rogamos, Senhor, que prepareis as nossas ações com a vossa inspiração, e as acompanheis com a vossa ajuda, a fim de que todos os nossos trabalhos e orações em Vós comecem sempre e convosco acabem. Por Cristo, Senhor Nosso. Amém.
\end{flushleft}
\begin{center}
    À Santíssima Virgem
\end{center}
\begin{flushleft}
    Ó Senhora minha, ó minha Mãe! Eu me ofereço todo a Vós, e, em prova da minha devoção para convosco, Vos consagro neste dia meus olhos, meus ouvidos, minha boca, meu coração e inteiramente todo o meu ser. E como assim sou vosso, ó boa Mãe, guardai-me e defendei-me como coisa e propriedade vossa. Amém.
\end{flushleft}
\begin{center}
    Ao Anjo da Guarda
\end{center}
\begin{flushleft}
    Santo Anjo do Senhor, meu zeloso guardador, se a ti me confiou a piedade divina, sempre me rege e guarda, governa e ilumina. Amém.
\end{flushleft}
\begin{center}
    Glória
\end{center}
\begin{flushleft}
    Glória ao Pai, e ao Filho e ao Espírito Santo. Assim como era no princípio, agora e sempre, por todos os séculos dos séculos. Amém.
\end{flushleft}
\newpage
\begin{center}
    \textbf{Orações do Meio-dia}
\end{center}
\begin{center}
    O Anjo do Senhor (Angelus) \\ \textcolor{VioletRed2}{\scriptsize{(Durante o ano)}}
\end{center}
\begin{flushleft}
    \VbarRed{} O Anjo do Senhor anunciou a Maria. \\
    \RbarRed{} E Ela concebeu do Espírito Santo. \\
    \VbarRed{} Ave, Maria, cheia de graça, o Senhor é convosco, bendita sois Vós entre as mulheres e bendito é o fruto do vosso ventre, Jesus. \\
    \RbarRed{} Santa Maria, Mãe de Deus, rogai por nós, pecadores, agora e na hora da nossa morte. Amém.
    \vspace{.2cm} \\
    \VbarRed{} Eis aqui a escrava do Senhor. \\
    \RbarRed{} Faça-se em mim segundo a vossa palavra. \\
    \VbarRed{} Ave, Maria, cheia de graça, o Senhor é convosco, bendita sois Vós entre as mulheres e bendito é o fruto do vosso ventre, Jesus. \\
    \RbarRed{} Santa Maria, Mãe de Deus, rogai por nós, pecadores, agora e na hora da nossa morte. Amém.
    \vspace{.2cm} \\
    \VbarRed{} E o Verbo de Deus se fez carne. \\
    \RbarRed{} E habitou entre nós. \\
    \VbarRed{} Ave, Maria, cheia de graça, o Senhor é convosco, bendita sois Vós entre as mulheres e bendito é o fruto do vosso ventre, Jesus. \\
    \RbarRed{} Santa Maria, Mãe de Deus, rogai por nós, pecadores, agora e na hora da nossa morte. Amém.
\end{flushleft}
\begin{center}
    Glória \\ \textcolor{VioletRed2}{\scriptsize{(3 Vezes)}}
\end{center}
\begin{flushleft}
    Glória ao Pai, e ao Filho e ao Espírito Santo. Assim como era no princípio, agora e sempre, por todos os séculos dos séculos. Amém.
\end{flushleft}
\newpage
\begin{center}
    Rainha do Céu (Regina Cæli) \\ \textcolor{VioletRed2}{\scriptsize{(Tempo Pascal)}}
\end{center}
\begin{flushleft}
    \VbarRed{} Rainha do Céu, alegrai-Vos, aleluia. \\
    \RbarRed{} Porque quem merecestes trazer em vosso seio, aleluia. \\
    \VbarRed{} Ressuscitou como disse, aleluia. \\
    \RbarRed{} Rogai a Deus por nós, aleluia.
    \vspace{.2cm} \\
    \VbarRed{} Exultai e alegrai-Vos, ó Virgem Maria, aleluia. \\
    \RbarRed{} Porque o Senhor ressuscitou verdadeiramente, aleluia.
    \vspace{.2cm} \\
    \textbf{\textit{Oremos:}} Ó Deus, que Vos dignastes alegrar o mundo com a Ressurreição do vosso Filho Jesus Cristo, Senhor Nosso, concedei-nos, Vos suplicamos, que por sua Mãe, a Virgem Maria, alcancemos as alegrias da vida eterna. Pelo mesmo Jesus Cristo, Senhor Nosso. Amém.
\end{flushleft}
\begin{center}
    Glória \\ \textcolor{VioletRed2}{\scriptsize{(3 Vezes)}}
\end{center}
\begin{flushleft}
    Glória ao Pai, e ao Filho e ao Espírito Santo. Assim como era no princípio, agora e sempre, por todos os séculos dos séculos. Amém.
\end{flushleft}
\newpage
\begin{center}
    \textbf{Orações da Noite - Completas}
\end{center}
\begin{center}
    Sinal da Cruz
\end{center}
\begin{flushleft}
    Pelo Sinal, \grecrossRed{} da Santa Cruz, livrai-nos Deus, \grecrossRed{} Nosso Senhor, dos nossos \grecrossRed{} inimigos. Em nome do Pai, \grecrossRed{} e do Filho, e do Espírito Santo. Amém.
\end{flushleft}
\begin{center}
    Oração da noite
\end{center}
\begin{flushleft}
    Ó Deus eterno, eu Vos adoro e dou graças por todos os vossos benefícios: porque me criaste e remiste por Jesus Cristo e me fizeste cristão e me esperastes, estando eu em pecado, e tantas vezes me tendes perdoado minhas culpas.
\end{flushleft}
\begin{center}
    Ato de Presença de Deus
\end{center}
\begin{flushleft}
    Meu Deus, dai-me luz para conhecer os pecados que hoje cometi, as suas causas e os meios de os evitar.
\end{flushleft}
\begin{center}
    Exame de Consciência
\end{center}
\begin{flushleft}
    Deveres para com Deus: \\ Comecei o dia oferecendo a Deus todos os meus pensamentos, palavras e ações? Fiz algumas outras orações durante o dia: agradecendo, pedindo, oferecendo a Deus o trabalho bem-feito, aceitando com fé os sofrimentos e contrariedades etc.? Procurei viver a fé e cumprir com os preceitos da Igreja? Caí em alguma prática supersticiosa? Fiz alguns pequenos sacrifícios ao comer, ao beber, nas conversas, na guarda da vista pela rua?
    \vspace{.2cm} \\
    Deveres para com o próximo: \\ Tratei os outros com compreensão e paciência? Manifestei-lhes aversão, desprezo ou irritação? Tive inveja? Falei mal da vida alheia, divulgando defeitos ou pecados dos outros? Fui egoísta, pensando só nas atenções que os outros deveriam dar-me, e esquecendo-me de ser prestativo, generoso e dedicado? Admiti sentimentos de ódio, rancor ou vingança? Magoei alguém com brincadeiras e comentários humilhantes? Procurei prestar pequenos serviços aos demais? Fiz o possível por auxiliar os que precisavam de ajuda material ou espiritual, sobretudo no trabalho e em casa? Rezei pelos outros e procurei aproximar algum amigo de Deus? \\
    \newpage
    Deveres para comigo: \\ Esforcei-me por melhorar hoje em alguma virtude, especialmente naquelas em que tenho mais dificuldades? Procurei cumprir com perfeição os meus deveres familiares e profissionais, lutando contra a preguiça, o desleixo, a desordem, o adiamento? Evitei pensamentos, palavras ou atos de orgulho, vaidade, preguiça, sensualidade ou avareza? Deixei-me arrastar pela curiosidade sensual e por desejos impuros? Fui insincero?
    \vspace{.2cm} \\
    Convém lembrar sempre que todos os dias temos de: \\ Glorificar a Deus. Imitar a Jesus Cristo. Invocar a Virgem Santíssima. Implorar a todos os Santos. Salvar a alma. Mortificar o corpo. Adquirir virtudes. Examinar a consciência. Expiar pecados. Evitar o inferno. Ganhar o Paraíso. Preparar a eternidade. Aproveitar o tempo. Edificar o próximo. Preterir o mundo. Combater os demônios. Dominar as paixões. Suportar a morte. Esperar o juízo.
\end{flushleft}
\begin{center}
    Ato de Fé
\end{center}
\begin{flushleft}
    Eu creio firmemente que há um só Deus em três Pessoas realmente distintas, Pai, Filho e Espírito Santo; que dá o Céu aos bons e o inferno aos maus para sempre. Creio que o Filho de Deus se fez homem e morreu na Cruz para nos salvar, e que ao terceiro dia ressuscitou. Creio tudo o mais que crê e ensina a Santa Igreja Católica, Apostólica, Romana, porque Deus, verdade infalível, lho revelou. E nesta crença quero viver e morrer.
\end{flushleft}
\begin{center}
    Ato de Esperança
\end{center}
\begin{flushleft}
    Eu espero, meu Deus, com firme confiança, que pelos merecimentos do meu Senhor Jesus Cristo me dareis a salvação eterna e as graças necessárias para consegui-la, porque Vós, sumamente bom e poderoso, o haveis prometido a quem observar fielmente os vossos mandamentos, como eu me proponho fazer com o vosso auxílio.
\end{flushleft}
\begin{center}
    Ato de Caridade
\end{center}
\begin{flushleft}
    Eu Vos amo, meu Deus, de todo o meu coração e sobre todas as coisas, porque sois infinitamente bom e amável, e antes quero perder tudo do que Vos ofender. Por amor de Vós amo o meu próximo como a mim mesmo.
\end{flushleft}
\newpage
\begin{center}
    Ato de Contrição
\end{center}
\begin{flushleft}
    Eu pecador, me confesso a Deus todo-poderoso, à Bem-aventurada sempre Virgem Maria, ao Bem-aventurado São Miguel Arcanjo, ao Bem-aventurado São João Batista, aos Santos Apóstolos São Pedro e São Paulo, a todos os Santos, e a vós irmãos, porque pequei muitas vezes por pensamentos, palavras e obras, por minha culpa, por minha culpa, minha máxima culpa. Portanto peço, à Bem-aventurada sempre Virgem Maria, ao Bem-aventurado São Miguel Arcanjo, ao Bem-aventurado São João Batista, aos Santos Apóstolos São Pedro e São Paulo, a todos os Santos, e a vós, irmãos, que roqueis por mim a Deus Nosso Senhor. Amém.
\end{flushleft}
\begin{center}
    Oração Particular \\ \textcolor{VioletRed2}{\scriptsize{(Fazer oração particular)}}
\end{center}
\begin{center}
    Jaculatória
\end{center}
\begin{flushleft}
    Dignai-Vos, Senhor, retribuir com a vida eterna a todos os que nos fazem bem por amor do vosso nome.
\end{flushleft}
\begin{center}
    Glória
\end{center}
\begin{flushleft}
    Glória ao Pai, e ao Filho e ao Espírito Santo. Assim como era no princípio, agora e sempre, por todos os séculos dos séculos. Amém.
\end{flushleft}
\newpage
\begin{center}
    \textbf{Meditação}
\end{center}
\begin{center}
    Vinde, Espírito Santo
\end{center}
\begin{flushleft}
    Vinde, Espírito Santo, enchei os corações dos vossos fiéis e acendei neles o fogo do vosso amor. \\
    \VbarRed{} Enviai o vosso Espírito e tudo será criado. \\
    \RbarRed{} E renovareis a face da terra.
    \vspace{.2cm} \\
    \textbf{\textit{Oremos:}} Ó Deus, que instruístes os corações dos vossos fiéis com a luz do Espírito Santo, concedei-nos amar, no mesmo Espírito, o que é reto, e gozar sempre a sua consolação. Por Cristo, Senhor Nosso. Amém.
    \vspace{.2cm} \\
    \textit{Antes:} \\ Meu Senhor e Meu Deus, creio firmemente que estás aqui, que me vês, que me ouves. Adoro-Te com profunda reverência. Peço-Te perdão dos meus pecados e graça para fazer com fruto este tempo de oração. \\ Minha Mãe Imaculada, São José, meu Pai e Senhor, meu Anjo da Guarda, intercedei por mim.
    \vspace{.2cm} \\
    \textit{Depois:} \\ Dou-Te graças, Meu Deus, pelos bons propósitos, afetos e inspirações que me comunicaste nesta meditação; peço-Te ajuda para os pôs em prática. \\ Minha Mãe Imaculada, São José, meu Pai e Senhor, meu Anjo da Guarda, intercedei por mim.
\end{flushleft}
\newpage
\begin{center}
    \textbf{Bênção dos Alimentos}
\end{center}
\begin{center}
    Bênção antes das refeições
\end{center}
\begin{flushleft}
    Abençoai-nos, Senhor, a nós e a estes dons que da vossa liberalidade recebemos. Por Cristo, Senhor Nosso. Amém.
    \vspace{.2cm} \\
    \textit{Almoço:} \\
    \VbarRed{} Que o Rei da eterna glória nos faça participantes da mesa celestial. \\
    \RbarRed{} Amém.
    \vspace{.2cm} \\
    \textit{Jantar:} \\
    \VbarRed{} Que o Rei da eterna glória nos conduza à Ceia da vida eterna. \\
    \RbarRed{} Amém.
    \vspace{.2cm} \\
    Em nome do Pai, \grecrossRed{} e do Filho, e do Espírito Santo. Amém.
\end{flushleft}
\begin{center}
    Benção depois das refeições
\end{center}
\begin{flushleft}
    Nós Vos damos graças, Deus onipotente, por todos vossos benefícios, Vós que viveis e reinais por todos os séculos dos séculos. Amém.
    \vspace{.2cm} \\
    \VbarRed{} Que Deus nos dê a sua paz. \\
    \RbarRed{} E a vida eterna. Amém.
    \vspace{.2cm} \\
    Em nome do Pai, \grecrossRed{} e do Filho, e do Espírito Santo. Amém.
\end{flushleft}
\newpage
\begin{center}
    \textbf{Santo Rosário}
\end{center}
\begin{center}
    Sinal da Cruz
\end{center}
\begin{flushleft}
    Pelo Sinal, \grecrossRed{} da Santa Cruz, livrai-nos Deus, \grecrossRed{} Nosso Senhor, dos nossos \grecrossRed{} inimigos. Em nome do Pai, \grecrossRed{} e do Filho, e do Espírito Santo. Amém.
\end{flushleft}
\begin{center}
    Jaculatórias
\end{center}
\begin{flushleft}
    \VbarRed{} Abri os meus lábios, ó Senhor. \\
    \RbarRed{} E minha boca anunciará vosso louvor.
    \vspace{.2cm} \\
    \VbarRed{} Vinde, ó Deus em meu auxílio. \\
    \RbarRed{} Socorrei-me sem demora.
\end{flushleft}
\begin{center}
    Vinde, Espírito Santo
\end{center}
\begin{flushleft}
    Vinde, Espírito Santo, enchei os corações dos vossos fiéis e acendei neles o fogo do vosso amor. \\
    \VbarRed{} Enviai o vosso Espírito e tudo será criado. \\
    \RbarRed{} E renovareis a face da terra.
    \vspace{.2cm} \\
    \textbf{\textit{Oremos:}} Ó Deus, que instruístes os corações dos vossos fiéis com a luz do Espírito Santo, concedei-nos amar, no mesmo Espírito, o que é reto, e gozar sempre a sua consolação. Por Cristo, Senhor Nosso. Amém. \\
\end{flushleft}
\begin{center}
    Oferecimento
\end{center}
\begin{flushleft}
    Divino Jesus, eu vos ofereço este rosário que vou rezar, contemplando os mistérios da nossa redenção. Concedei-me, pela intercessão de Maria, Vossa Mãe Santíssima, a quem me dirijo, as virtudes que me são necessárias para rezá-lo bem e a graça de ganhar as indulgências anexas a esta santa devoção.
\end{flushleft}
\begin{center}
    Jaculatória
\end{center}
\begin{flushleft}
    \VbarRed{} Dignai-vos, ó Virgem Sagrada, consentir em ser por mim louvada. \\
    \RbarRed{} Dai-me força contra os vossos inimigos.
\end{flushleft}
\begin{center}
    Credo
\end{center}
\begin{flushleft}
    Creio em Deus Pai Todo-poderoso, criador do céu e da terra; e em Jesus Cristo, seu único Filho, Nosso Senhor, que foi concebido pelo poder do Espírito Santo; nasceu da Virgem Maria, padeceu sob Pôncio Pilatos, foi crucificado, morto e sepultado; desceu à mansão dos mortos; ressuscitou ao terceiro dia; subiu aos céus, está sentado à direita de Deus Pai Todo-poderoso, donde há de vir a julgar os vivos e os mortos; creio no Espírito Santo, na santa Igreja Católica, na comunhão dos santos, na remissão dos pecados, na ressurreição da carne, na vida eterna. Amém.
\end{flushleft}
\begin{center}
    Pai Nosso
\end{center}
\begin{flushleft}
    Pai nosso que estais nos céus, santificado seja o vosso nome; venha a nós o vosso reino, seja feita a vossa vontade assim na terra como no céu. O pão nosso de cada dia nos daí hoje; perdoai-nos as nossas ofensas, assim como nós perdoamos a quem nos tem ofendido, e não nos deixeis cair em tentação, mas livrai-nos do mal. Amém.
\end{flushleft}
\begin{center}
    Ave Maria \\ \textcolor{VioletRed2}{\scriptsize{(3 Vezes)}}
\end{center}
\begin{flushleft}
    Ave, Maria, cheia de graça, o Senhor é convosco, bendita sois Vós entre as mulheres e bendito é o fruto do vosso ventre, Jesus. Santa Maria, Mãe de Deus, rogai por nós, pecadores, agora e na hora da nossa morte. Amém.
\end{flushleft}
\begin{center}
    Glória
\end{center}
\begin{flushleft}
    Glória ao Pai, e ao Filho e ao Espírito Santo. Assim como era no princípio, agora e sempre, por todos os séculos dos séculos. Amém.
    \vspace{.2cm} \\
    \VbarRed{} Ó meu Jesus, perdoai-nos, livrai-nos do fogo do inferno. \\
    \RbarRed{} Levai as almas todas para o Céu e socorrei principalmente as que mais precisarem.
    \vspace{.2cm} \\
    \VbarRed{} Ó Maria concebida sem pecado.\\
    \RbarRed{} Rogai por nós, que recorremos a Vós.
\end{flushleft}
\newpage
\begin{center}
    Mistérios Gozosos \\ \textcolor{VioletRed2}{\scriptsize{(Segunda-Feira, Quinta-Feira, Domingos do Advento e Natal)}} \\
    \hfill{} \break{}
    1\textordmasculine{} Mistério \\ Anunciação do Anjo e a Encarnação do Verbo no seio Puríssimo da Virgem Maria
\end{center}
\begin{flushleft}
    Do Evangelho segundo São Lucas - (\textcolor{VioletRed2}{Lc 1,26-38}). \\
    \hfill{} \break{}
    Quando Isabel estava no sexto mês, o anjo Gabriel foi enviado por deus a uma cidade da Galileia, chamada Nazaré, a uma virgem prometida em casamento a um homem de nome José, da casa de Davi. O nome da virgem era Maria. O anjo entrou onde ela estava e disse: ``Alegra-te, cheia de graça! O Senhor está contigo''. Ela perturbou-se com essas palavras e pôs-se a pensar no que significaria a saudação. O anjo, então, disse: ``Não temas, Maria! Encontraste graça junto a Deus. Conceberás e darás à luz um filho, e lhe porás o nome de Jesus. Ele será grande e será chamado Filho do Altíssimo, e o Senhor Deus lhe dará o trono de seu pai Davi. Ele reinará para sempre sobre a casa de Jacó, e o seu reino não terá fim''.
    \vspace{.2cm} \\
    Maria, então, perguntou ao anjo: ``Como acontecerá isso, se não conheço homem algum?'' O anjo respondeu: ``O Espírito Santo descerá sobre ti, e o poder do Altíssimo te cobrirá com sua sombra. Por isso, aquele que vai nascer é santo e será chamado Filho de Deus. Também Isabel, tua parenta, concebeu um filho na sua velhice, já está no sexto mês aquela que era chamada estéril, pois \textit{para Deus nada é impossível}''. Então Maria disse: ``Eis aqui a serva do Senhor! Faça-se em mim segundo a tua palavra''. E o anjo saiu da sua presença. \\
    \hfill{} \break{}
    Pai Nosso, dez Ave Maria, Glória.
\end{flushleft}
\newpage
\begin{center}
    2\textordmasculine{} Mistério \\ Visitação da Virgem Maria a sua prima Santa Isabel e Santificação de São João Batista
\end{center}
\begin{flushleft}
    Do Evangelho segundo São Lucas - (\textcolor{VioletRed2}{Lc 1,39-42}). \\
    \hfill{} \break{}
    Naqueles dias, Maria levantou-se e foi apressadamente à região montanhosa, a uma cidade de Judá. Ela entrou na casa de Zacarias e saudou Isabel. Quando Isabel ouviu a saudação de Maria, a criança saltou de alegria em seu ventre. Isabel ficou repleta de Espírito Santo e, com voz forte, exclamou: ``Bendita és tu entre as mulheres e bendito é o fruto do teu ventre! Como me acontece que a mãe do meu Senhor venha a mim? Logo que ressoou aos meus ouvidos a tua saudação, a criança pulou de alegria no meu ventre. Bem-aventurada aquela que acreditou, porque se cumprirá o que lhe foi dito da parte do Senhor''. \\
    \hfill{} \break{}
    Pai Nosso, dez Ave Maria, Glória.
\end{flushleft}
\begin{center}
    3\textordmasculine{} Mistério \\ Nascimento do Menino Jesus em Belém
\end{center}
\begin{flushleft}
    Do Evangelho segundo São Lucas - (\textcolor{VioletRed2}{Lc 2,1-7}). \\
    \hfill{} \break{}
    Naqueles dias foi publicado um decreto do imperador Augusto ordenando o recenseamento do mundo inteiro. Esse primeiro recenseamento aconteceu quando Quirino era governador da Síria. Todos iam registrar-se, cada um em sua própria cidade. Também José - que era da casa e da linhagem de Davi - subiu da Galileia, da cidade de Nazaré, à Judeia, à cidade de Davi, chamada Belém, para registrar-se com Maria, sua esposa, que estava grávida. Quando estavam ali, completaram-se os dias de ela dar à luz. Ela deu à luz o seu filho, o primogênito, envolveu-o em faixas e deitou-o numa manjedoura, porque não havia lugar para eles na hospedaria. \\
    \hfill{} \break{}
    Pai Nosso, dez Ave Maria, Glória.
\end{flushleft}
\newpage
\begin{center}
    4\textordmasculine{} Mistério \\ Apresentação do Menino Jesus no templo e a Purificação da Virgem Maria
\end{center}
\begin{flushleft}
    Do Evangelho segundo São Lucas - (\textcolor{VioletRed2}{Lc 2,21-24}). \\
    \hfill{} \break{}
    Completados os oitos dias para a circuncisão do menino, deram-lhe o nome de Jesus, como fora chamado pelo anjo, antes de ser concebido no ventre de sua mãe.
    \vspace{.2cm} \\
    Quando se completaram os dias da purificação, segundo a lei de Moisés, levaram o menino a Jerusalém para apresentá-lo ao Senhor, conforme está escrito na Lei do Senhor: ``\textit{Todo primogênito masculino será consagrado ao Senhor}''; e para oferecer em sacrifício \textit{um par de rolas ou dois pombinhos}, como está escrito na Lei do Senhor. \\
    \hfill{} \break{}
    Pai Nosso, dez Ave Maria, Glória.
\end{flushleft}
\begin{center}
    5\textordmasculine{} Mistério \\ Perda e o Encontro do Menino Jesus no Templo discutindo com os doutores da Lei
\end{center}
\begin{flushleft}
    Do Evangelho segundo São Lucas - (\textcolor{VioletRed2}{Lc 2,41-50}). \\
    \hfill{} \break{}
    Todos os anos, os pais de Jesus iam a Jerusalém para a festa da Páscoa. Quando ele completou doze anos, subiram para a festa, como de costume. Terminados os dias da festa, no momento de voltarem, Jesus permaneceu em Jerusalém, sem que seus pais percebessem. Pensando que se encontrasse na caravana, fizeram o caminho de um dia e procuravam-no entre os parentes e conhecidos; mas, como não o encontrassem, voltaram a Jerusalém, à procura dele. Depois de três dias o encontraram no templo, sentado entre os mestres ouvindo-os e fazendo-lhes perguntas. Todos os que ouviam o menino ficavam extasiados com sua inteligência e suas repostas. Quando o viram, seus pais ficaram admirados, e sua mãe lhe disse: ``Filho, por que agiste assim conosco? Olha, teu pai e eu andávamos, angustiados, à tua procura!'' Ele respondeu: ``Por que me procuráveis? Não sabeis que eu devo estar naquilo que é de meu Pai?'' Eles, porém, não entenderam o que ele lhes havia dito. \\
    \hfill{} \break{}
    Pai Nosso, dez Ave Maria, Glória.
\end{flushleft}
\newpage
\begin{center}
    Mistérios Dolorosos \\ \textcolor{VioletRed2}{\scriptsize{(Terça-Feira, Sexta-Feira, Domingos da Quaresma e Tríduo Pascal)}} \\
    \hfill{} \break{}
    1\textordmasculine{} Mistério \\ Agonia de Nosso Senhor Jesus Cristo e a oração no Jardim das Oliveiras
\end{center}
\begin{flushleft}
    Do Evangelho segundo São Marcos - (\textcolor{VioletRed2}{Mc 14,32-36}). \\
    \hfill{} \break{}
    Chegaram a um lugar chamado Getsêmani. Jesus disse aos discípulos: ``Sentai-vos aqui, enquanto eu vou orar''. Levou consigo Pedro, Tiago e João, e começou a sentir pavor e angústia. Ele disse-lhes: ``Minha alma está triste até a morte! Ficai aqui e vigiai''! Então foi um pouco mais adiante, caiu por terra e orava para que, se possível, passasse dele aquela hora. E dizia: ``Abá, Pai! Tudo te é possível. Afasta de mim este cálice! Contudo, não seja o que eu quero, mas o que tu queres''. \\
    \hfill{} \break{}
    Pai Nosso, dez Ave Maria, Glória.
\end{flushleft}
\begin{center}
    2\textordmasculine{} Mistério \\ Flagelação de Nosso Senhor Jesus Cristo
\end{center}
\begin{flushleft}
    Do Evangelho segundo São Mateus - (\textcolor{VioletRed2}{Mt 27,22-26}). \\
    \hfill{} \break{}
    Pilatos perguntou: ``Que farei, então, com Jesus, que é chamado o Cristo?'' Todos responderam: ``Seja crucificado!'' Pilatos falou: ``Mas, que mal ele fez?'' Eles, porém, gritaram com mais força: ``Seja crucificado!'' Quando Pilatos viu que nada conseguia e que, ao contrário, aumentava o tumulto, mandou trazer água, lavou as mãos diante da multidão e disse: ``Sou inocente do sangue deste homem. A responsabilidade é vossa!'' O povo todo respondeu: ``Que o sangue dele recaia sobre nós e sobre nossos filhos''. Então Pilatos soltou Barrabás, mandou flagelar Jesus e entregou-o para ser crucificado. \\
    \hfill{} \break{}
    Pai Nosso, dez Ave Maria, Glória.
\end{flushleft}
\newpage
\begin{center}
    3\textordmasculine{} Mistério \\ Nosso Senhor Jesus Cristo é coroado de espinhos
\end{center}
\begin{flushleft}
    Do Evangelho segundo São Mateus - (\textcolor{VioletRed2}{Mt 27,27-31}). \\
    \hfill{} \break{}
    Em seguida, os soldados do governador levaram Jesus ao pretório e reuniram toda a guarnição em volta dele. Tiraram-lhe as vestes e vestiram-no com um manto escarlate. Depois, puseram-lhe na cabeça uma coroa de espinhos que trançaram, e um caniço na mão direita, e ajoelharam-se diante de Jesus, enquanto diziam, zombando: ``Salve, rei dos judeus!'' Cuspiram nele e bateram-lhe na cabeça com o caniço. Depois de zombar dele, tiraram-lhe o manto e o vestiram com suas próprias vestes. \\
    \hfill{} \break{}
    Pai Nosso, dez Ave Maria, Glória.
\end{flushleft}
\begin{center}
    4\textordmasculine{} Mistério \\ Nosso Senhor Jesus Cristo carrega a Cruz nas costas a caminho do Calvário
\end{center}
\begin{flushleft}
    Do Evangelho segundo São João - (\textcolor{VioletRed2}{Jo 19,16-22}). \\
    \hfill{} \break{}
    Pilatos, então, lhes entregou Jesus para ser crucificado.
    \vspace{.2cm} \\
    Então receberam Jesus. E, carregando ele próprio sua cruz, saiu para o lugar chamado Calvário, em hebraico: Gólgota. Lá, eles o crucificaram com outros dois, um de cada lado, e Jesus no meio. Pilatos mandou escrever e afixar na cruz um letreiro; estava escrito assim: ``Jesus Nazareno, o Rei dos Judeus''. Como o lugar onde Jesus fora crucificado, era perto da cidade, muitos judeus leram o letreiro, que estava escrito em hebraico, latim e grego. Os chefes dos sacerdotes dos judeus disseram então a Pilatos: ``Não escrevas: `O Rei dos Judeus' e sim: `Ele disse: Eu sou o Rei dos Judeus'\,''. Pilatos respondeu: ``O que escrevi, escrevi''. \\
    \hfill{} \break{}
    Pai Nosso, dez Ave Maria, Glória.
\end{flushleft}
\newpage
\begin{center}
    5\textordmasculine{} Mistério \\ Crucifixão e Morte de Nosso Senhor Jesus Cristo
\end{center}
\begin{flushleft}
    Do Evangelho segundo São João - (\textcolor{VioletRed2}{Jo 19,23-30}). \\
    \hfill{} \break{}
    Depois de crucificarem Jesus, os soldados pegaram suas vestes e as dividiram em quatro partes, para cada soldado uma parte. Pegaram também a túnica, que era feita sem costura, uma peça única, de alto a baixo, e combinaram: ``Não vamos rasgar a túnica, vamos tirar a sorte para ver de quem será''. Assim cumpriu-se a Escritura:
    \vspace{.2cm} \\
    ``\textit{Repartiram entre si as minhas vestes e tiraram a sorte sobre minha túnica}''. Foi o que os soldados fizeram.
    \vspace{.2cm} \\
    Junto à cruz de Jesus estavam de pé sua mãe e a irmã de sua mãe, Maria de Cléofas, e Maria Madalena. Jesus, ao ver sua mãe e, ao lado dela, o discípulo a quem amava, disse à mãe: ``Mulher, eis o teu filho!'' Depois disse ao discípulo: ``Eis tua mãe!'' A partir daquela hora, o discípulo a acolheu em sua casa.
    \vspace{.2cm} \\
    Em seguida, sabendo Jesus que tudo estava consumado, para que se cumprisse a Escritura, disse: ``Tenho sede''! Havia ali uma vasilha cheia de vinagre. Fixaram uma esponja embebida em vinagre num ramo de hissopo e a levaram à sua boca. Depois que tomou o vinagre, ele disse: ``Está consumado''. E, inclinando a cabeça, entregou o espírito. \\
    \hfill{} \break{}
    Pai Nosso, dez Ave Maria, Glória.
\end{flushleft}
\newpage
\begin{center}
    Mistérios Gloriosos \\ \textcolor{VioletRed2}{\scriptsize{(Quarta-Feira, Sábado, Domingos da Páscoa e Tempo Comum)}} \\
    \hfill{} \break{}
    1\textordmasculine{} Mistério \\ Ressurreição de Nosso Senhor Jesus Cristo
\end{center}
\begin{flushleft}
    Do Evangelho segundo São Marcos - (\textcolor{VioletRed2}{Mc 16,9-15}). \\
    \hfill{} \break{}
    Tendo ressuscitado, na madrugada do primeiro dia da semana, Jesus apareceu primeiro a Maria Madalena, de quem tinha expulsado sete demônios. Ela foi anunciar o fato aos que acompanharam Jesus, e que estavam aflitos e choravam. Quando ouviram que ele estava vivo e havia sido visto por ela, não acreditaram. Depois disso, Jesus, de aspecto mudado, apareceu a dois deles, enquanto estavam a caminho do campo. Estes voltaram para anunciá-lo aos outros, os quais tampouco acreditaram.
    \vspace{.2cm} \\
    Por fim, Jesus apareceu aos onze discípulos, enquanto à mesa. Ele os censurou por sua falta de fé e sua dureza de coração, por não terem acreditado naqueles que o tinham visto ressuscitado. Então disse-lhes: ``Ide pelo mundo inteiro e proclamai o Evangelho a toda criatura!
    \vspace{.2cm} \\
    Quem crer e for batizado será salvo, mas quem não crer, será condenado. Eis os sinais que acompanharão aqueles que crerem: expulsarão demônios em meu nome; falarão novas línguas; se pegarem em serpentes e beberem veneno mortal, não lhes fará mal algum; e quando impuserem as mãos sobre os enfermos, estes ficarão curados''. \\
    \hfill{} \break{}
    Pai Nosso, dez Ave Maria, Glória.
\end{flushleft}
\begin{center}
    2\textordmasculine{} Mistério \\ Ascensão de Nosso Senhor Jesus Cristo
\end{center}
\begin{flushleft}
    Do Evangelho segundo São Marcos - (\textcolor{VioletRed2}{Mc 16,19-20}). \\
    \hfill{} \break{}
    Depois de falar com os discípulos, o Senhor Jesus foi elevado ao céu e sentou-se à direita de Deus.
    \vspace{.2cm} \\
    Então, os discípulos foram anunciar por toda parte. O Senhor cooperava, confirmando a palavra pelos sinais que a acompanhavam. \\
    \hfill{} \break{}
    Pai Nosso, dez Ave Maria, Glória.
\end{flushleft}
\newpage
\begin{center}
    3\textordmasculine{} Mistério \\ Descida do Espírito Santo sobre a Virgem Maria e os Apóstolos reunidos no Cenáculo
\end{center}
\begin{flushleft}
    Do Evangelho segundo São João - (\textcolor{VioletRed2}{Jo 14,25-31}). \\
    \hfill{} \break{}
    Eu vos tenho falado dessas coisas estando ainda convosco. Ora, o Paráclito, o Espírito Santo que o Pai enviará em meu nome, ele vos ensinará tudo e vos recordará tudo o que eu vos tenho dito. Deixo-vos a paz, dou-vos a minha paz. Eu não a dou, como a dá o mundo. Não se perturbe, nem se atemorize o vosso coração. Ouvistes o que vos disse: `Eu vou, mas voltarei a vós'. Se me amásseis, ficaríeis alegres porque vou para o Pai, pois o Pai é maior do que eu.
    \vspace{.2cm} \\
    Disse-vos isso agora, antes que aconteça, para que, quando acontecer, creiais. Já não falarei mais convosco, pois vem aí o príncipe deste mundo. Ele nada pode contra mim, mas é preciso que o mundo saiba que eu amo o Pai, e faço como o Pai me mandou. Levantai-vos! Vamo-nos daqui!'' \\
    \hfill{} \break{}
    Pai Nosso, dez Ave Maria, Glória.
\end{flushleft}
\begin{center}
    4\textordmasculine{} Mistério \\ Assunção de Nossa Senhora aos Céus de corpo e alma
\end{center}
\begin{flushleft}
    Do Evangelho segundo São Lucas - (\textcolor{VioletRed2}{Lc 11,27-28}). \\
    \hfill{} \break{}
    Enquanto Jesus assim falava, uma mulher levantou a voz, do meio da multidão, e lhe disse: ``Bem-aventurado o ventre que te gerou e os seios que te amamentaram''. Ele respondeu: ``Bem-aventurados, antes, os que ouvem a Palavra de Deus e a guardam''. \\
    \hfill{} \break{}
    Pai Nosso, dez Ave Maria, Glória.
\end{flushleft}
\newpage
\begin{center}
    5\textordmasculine{} Mistério \\ Coroação de Nossa Senhora como Rainha do Céu e da Terra
\end{center}
\begin{flushleft}
    Do Evangelho segundo São Lucas - (\textcolor{VioletRed2}{Lc 1,46-55}). \\
    \hfill{} \break{}
    ``\textit{A minha alma engrandece o Senhor}, e meu espírito exulta \textit{em Deus, meu Salvador}, porque olhou \textit{para a condição humilde de sua serva.} Todas as gerações, desde agora, me chamarão bem-aventurada, porque o Poderoso fez por mim grandes coisas. Santo é o seu nome, e sua misericórdia se estende, de geração em geração, sobre aqueles que o temem. Ele manifestou poder com o seu braço: dispersou os soberbos nos pensamentos de seu coração. Depôs os poderosos de seus tronos e exaltou os de condição humilde. Encheu de bens os famintos e despediu os ricos sem nada. Amparou Israel, seu servo, lembrando-se da misericórdia, como prometera a nossos pais, a Abraão e à sua descendência, para sempre. \\
    \hfill{} \break{}
    Pai Nosso, dez Ave Maria, Glória.
\end{flushleft}
\begin{center}
    Agradecimento
\end{center}
\begin{flushleft}
    Infinitas graças vos damos, soberana Rainha, pelos benefícios que todos os dias recebemos das vossas mãos liberais. Dignai-vos, agora e sempre, tomar-nos debaixo do vosso poderoso amparo, e para mais vos obrigar, vos saudamos com uma Salve Rainha.
\end{flushleft}
\begin{center}
    Salve Rainha
\end{center}
\begin{flushleft}
    Salve Rainha, Mãe de misericórdia, vida, doçura, esperança nossa, salve! A vós bradamos, os degredados filhos de Eva. A Vós suspiramos, gemendo e chorando neste vale de lágrimas. Eia, pois advogada nossa, esses vossos olhos misericordiosos a nós volvei, e depois deste desterro mostrai-nos Jesus, bendito fruto do vosso ventre. Ó clemente, ó piedosa, ó doce sempre Virgem Maria.
\end{flushleft}
\newpage
\begin{center}
    Ladainha Lauretana
\end{center}
\begin{flushleft}
    \VbarRed{} Senhor, tende piedade de nós. \\
    \RbarRed{} Senhor, tende piedade de nós. \\
    \VbarRed{} Cristo, tende piedade de nós. \\
    \RbarRed{} Cristo, tende piedade de nós. \\
    \VbarRed{} Senhor, tende piedade de nós. \\
    \RbarRed{} Senhor, tende piedade de nós.
    \vspace{.2cm} \\
    \VbarRed{} Cristo, ouvi-nos. \\
    \RbarRed{} Cristo, ouvi-nos. \\
    \VbarRed{} Cristo, atendei-nos. \\
    \RbarRed{} Cristo, atendei-nos.
    \vspace{.2cm} \\
    \VbarRed{} Deus pai do Céu. \\
    \RbarRed{} Tende piedade de nós. \\
    \VbarRed{} Deus Filho Redentor do mundo. \\
    \RbarRed{} Tende piedade de nós. \\
    \VbarRed{} Deus Espírito Santo. \\
    \RbarRed{} Tende piedade de nós. \\
    \VbarRed{} Santíssima Trindade, que sois um só Deus. \\
    \RbarRed{} Tende piedade de nós.
    \vspace{.2cm} \\
    \VbarRed{} Santa Maria. \\
    \RbarRed{} Rogai por nós.
    \vspace{.2cm} \\
    \textcolor{VioletRed2}{\small{(Repete-se a mesma resposta a partir daqui)}}
    \vspace{.2cm} \\
    \VbarRed{} Santa Mãe de Deus, \\
    \VbarRed{} Santa Virgem das virgens, \\
    \VbarRed{} Mãe de Cristo, \\
    \VbarRed{} Mãe da Igreja, \\
    \VbarRed{} Mãe da misericórdia, \\
    \VbarRed{} Mãe da divina graça, \\
    \VbarRed{} Mãe da esperança, \\
    \VbarRed{} Mãe puríssima, \\
    \VbarRed{} Mãe castíssima, \\
    \VbarRed{} Mãe sempre virgem, \\
    \VbarRed{} Mãe imaculada, \\
    \VbarRed{} Mãe digna de amor, \\
    \VbarRed{} Mãe admirável, \\
    \VbarRed{} Mãe do bom conselho, \\
    \VbarRed{} Mãe do Criador, \\
    \VbarRed{} Mãe do Salvador, \\
    \VbarRed{} Virgem prudentíssima, \\
    \VbarRed{} Virgem venerável, \\
    \VbarRed{} Virgem louvável, \\
    \VbarRed{} Virgem poderosa, \\
    \VbarRed{} Virgem clemente, \\
    \VbarRed{} Virgem fiel, \\
    \VbarRed{} Espelho da perfeição, \\
    \VbarRed{} Sede da sabedoria, \\
    \VbarRed{} Fonte da nossa alegria, \\
    \VbarRed{} Vaso espiritual, \\
    \VbarRed{} Tabernáculo da eterna glória, \\
    \VbarRed{} Moradia consagrada a Deus, \\
    \VbarRed{} Rosa mística, \\
    \VbarRed{} Torre de Davi, \\
    \VbarRed{} Torre de marfim, \\
    \VbarRed{} Casa de ouro, \\
    \VbarRed{} Arca da aliança, \\
    \VbarRed{} Porta do Céu, \\
    \VbarRed{} Estrela da manhã, \\
    \VbarRed{} Saúde dos enfermos, \\
    \VbarRed{} Refúgio dos pecadores, \\
    \VbarRed{} Socorro dos migrantes, \\
    \VbarRed{} Consoladora dos aflitos, \\
    \VbarRed{} Auxílio dos cristãos, \\
    \VbarRed{} Rainha dos Anjos, \\
    \VbarRed{} Rainha dos Patriarcas, \\
    \VbarRed{} Rainha dos Profetas, \\
    \VbarRed{} Rainha dos Apóstolos, \\
    \VbarRed{} Rainha dos Mártires, \\
    \VbarRed{} Rainha dos Confessores da fé, \\
    \VbarRed{} Rainha das Virgens, \\
    \VbarRed{} Rainha de todos os Santos, \\
    \VbarRed{} Rainha concebida sem pecado original, \\
    \VbarRed{} Rainha assunta aos Céus, \\
    \VbarRed{} Rainha do Santo Rosário, \\
    \VbarRed{} Rainha da Paz.
    \vspace{.2cm} \\
    \VbarRed{} Cordeiro de Deus, que tirais o pecado do mundo. \\
    \RbarRed{} Perdoai-nos, Senhor. \\
    \VbarRed{} Cordeiro de Deus, que tirais o pecado do mundo. \\
    \RbarRed{} Ouvi-nos, Senhor. \\
    \VbarRed{} Cordeiro de Deus, que tirais o pecado do mundo. \\
    \RbarRed{} Tende piedade de nós.
    \newpage
    À Vossa proteção nos acolhemos, Santa Mãe de Deus; não desprezeis as súplicas que em nossas necessidades Vos dirigimos, mas livrai-nos sempre de todos os perigos, ó Virgem gloriosa e bendita.
    \vspace{.2cm} \\
    \VbarRed{} Rogai por nós, Santa Mãe de Deus. \\
    \RbarRed{} Para que sejamos, dignos das promessas de Cristo.
    \vspace{.2cm} \\
    \textbf{\textit{Oremos:}} Infundi, Senhor, nós Vos pedimos, em nossas almas a vossa graça, para que nós, que conhecemos pela Anunciação do Anjo a Encarnação de Jesus Cristo, vosso filho, cheguemos por sua Paixão e sua Cruz à glória da Ressurreição. Pelo mesmo Jesus Cristo, Senhor Nosso.
    \vspace{.2cm} \\
    Em nome do Pai, \grecrossRed{} e do Filho, e do Espírito Santo. Amém.
\end{flushleft}
\newpage
\begin{center}
    \textbf{Visita ao Santíssimo Sacramento}
\end{center}
\begin{center}
    Sinal da Cruz
\end{center}
\begin{flushleft}
    Pelo Sinal, \grecrossRed{} da Santa Cruz, livrai-nos Deus, \grecrossRed{} Nosso Senhor, dos nossos \grecrossRed{} inimigos. Em nome do Pai, \grecrossRed{} e do Filho, e do Espírito Santo. Amém.
    \vspace{.2cm} \\
    \VbarRed{} Graças e louvores se deem a todo momento. \\
    \RbarRed{} Ao Santíssimo e Diviníssimo Sacramento.
\end{flushleft}
\begin{center}
    Ato de Contrição
\end{center}
\begin{flushleft}
    Eu pecador, me confesso a Deus todo-poderoso, à Bem-aventurada sempre Virgem Maria, ao Bem-aventurado São Miguel Arcanjo, ao Bem-aventurado São João Batista, aos Santos Apóstolos São Pedro e São Paulo, a todos os Santos, e a vós irmãos, porque pequei muitas vezes por pensamentos, palavras e obras, por minha culpa, por minha culpa, minha máxima culpa. Portanto peço, à Bem-aventurada sempre Virgem Maria, ao Bem-aventurado São Miguel Arcanjo, ao Bem-aventurado São João Batista, aos Santos Apóstolos São Pedro e São Paulo, a todos os Santos, e a vós, irmãos, que roqueis por mim a Deus Nosso Senhor. Amém.
\end{flushleft}
\begin{center}
    Pai Nosso
\end{center}
\begin{flushleft}
    Pai nosso que estais nos céus, santificado seja o vosso nome; venha a nós o vosso reino, seja feita a vossa vontade assim na terra como no céu. O pão nosso de cada dia nos daí hoje; perdoai-nos as nossas ofensas, assim como nós perdoamos a quem nos tem ofendido, e não nos deixeis cair em tentação, mas livrai-nos do mal. Amém.
\end{flushleft}
\begin{center}
    Ave Maria
\end{center}
\begin{flushleft}
    Ave, Maria, cheia de graça, o Senhor é convosco, bendita sois Vós entre as mulheres e bendito é o fruto do vosso ventre, Jesus. Santa Maria, Mãe de Deus, rogai por nós, pecadores, agora e na hora da nossa morte. Amém.
\end{flushleft}
\begin{center}
    Glória
\end{center}
\begin{flushleft}
    Glória ao Pai, e ao Filho e ao Espírito Santo. Assim como era no princípio, agora e sempre, por todos os séculos dos séculos. Amém.
\end{flushleft}
\begin{center}
    O Salutáris Hóstia
\end{center}
\begin{flushleft}
    Ó Hóstia salutar, que abres a porta do céu. Guerras hostis ameaçam-nos: dá-nos força e auxílio. Seja sempre dada glória ao Senhor Uno e Trino: que nos dê uma vida sem termo na Pátria.
\end{flushleft}
\begin{center}
    Ave Verum Corpus
\end{center}
\begin{flushleft}
    Ave, ó verdadeiro Corpo nascido da Virgem Maria. Que verdadeiramente padeceu e foi imolado na Cruz pelo homem. De seu lado transpassado fluiu água e sangue. Sede por nós o penhor no momento da morte. Ó doce Jesus! Ó bom Jesus! Ó Jesus filho de Maria.
\end{flushleft}
\begin{center}
    Adoro te Devote
\end{center}
\begin{flushleft}
    Adoro-Vos com devoção, Deus escondido, que sob estas aparências estais presente. A Vós se submete meu coração por inteiro, e ao contemplar-Vos se rende totalmente.
    \vspace{.2cm} \\
    A vista, o tato, o gosto sobre Vós se enganam, mas basta o ouvido para crer com firmeza. Creio em tudo o que disse o Filho de Deus; nada mais verdadeiro que esta palavra de verdade.
    \vspace{.2cm} \\
    Na Cruz estava oculta a divindade, mas aqui se esconde também a humanidade; creio, porém, e confesso uma e outra, e peço o que pediu o ladrão arrependido.
    \vspace{.2cm} \\
    Não vejo as chagas, como Tomé as viu, mas confesso que sois o meu Deus. Fazei que eu creia mais e mais em Vós, que em Vós espere, que Vos ame.
    \vspace{.2cm} \\
    Ó memorial da morte do Senhor! Ó Pão vivo que dais a vida a vida ao homem! Que a minha alma sempre de Vós viva, que sempre lhe seja doce o vosso sabor.
    \vspace{.2cm} \\
    Bom pelicano, Senhor Jesus! Limpai-me a mim, imundo, com o vosso Sangue, Sangue do qual uma só gota pode salvar do pecado o mundo inteiro.\
    \vspace{.2cm} \\
    Jesus, a quem agora contemplo escondido, rogo-Vos se cumpra o que tanto desejo: que, ao contemplar-Vos face a face, seja eu feliz vendo a vossa glória. Amém.
\end{flushleft}
\newpage
\begin{center}
    Tantum Ergo
\end{center}
\begin{flushleft}
    Tão grande sacramento humildemente adoremos. Da antiga Lei as figuras cedam ao novo rito. Sirva a fé de suplemento à fraqueza dos sentidos. Ao Pai e ao Filho, seja dado louvor e júbilo, saudação, honra, virtude assim como a bênção. Ao que de ambos procede [o Espírito Santo] os mesmos louvores demos. Amém.
    \vspace{.2cm} \\
    \VbarRed{} Do céu lhes deste o pão.\ (\textcolor{VioletRed2}{T.P.} Aleluia) \\
    \RbarRed{} Que contém todo sabor.\ (\textcolor{VioletRed2}{T.P.} Aleluia)
    \vspace{.2cm} \\
    \textbf{\textit{Oremos:}} Deus, que neste admirável sacramento nos deixastes o memorial da vossa Paixão, dai-nos venerar com tão grande amor o mistério do vosso Corpo e do vosso Sangue, que possamos colher continuamente os frutos da vossa Redenção. Vós que viveis e reinais por todos os séculos dos séculos. Amém.
    \vspace{.2cm} \\
    \textbf{\textit{Comunhão Espiritual:}} Meu Jesus, eu creio que estais presente no Santíssimo Sacramento. Amo-Vos sobre todas as coisas e minha alma suspira por Vós, mas, como não posso receber-Vos agora, no Santíssimo Sacramento, vinde ao menos espiritualmente a meu coração. Abraço-me convosco como se já estivésseis comigo; uno-me convosco inteiramente. Ah! Não permitais que eu algum dia me separe de Vós. Ó Jesus, sumo bem e doce amor meu, inflamai meu coração, a fim de que esteja para sempre abrasado em vosso Amor. Amém.
    \vspace{.2cm} \\
    Em nome do Pai, \grecrossRed{} e do Filho, e do Espírito Santo. Amém.
\end{flushleft}
\newpage
\begin{center}
    \textbf{Via Crucis}
\end{center}
\begin{center}
    Sinal da Cruz
\end{center}
\begin{flushleft}
    Pelo Sinal, \grecrossRed{} da Santa Cruz, livrai-nos Deus, \grecrossRed{} Nosso Senhor, dos nossos \grecrossRed{} inimigos. Em nome do Pai, \grecrossRed{} e do Filho, e do Espírito Santo. Amém.
    \vspace{.2cm} \\
    \textit{Oração diante do altar:}
    \vspace{.2cm} \\
    Em união com Maria, a Mãe das dores, vamos, ó Jesus, percorrer o caminho doloroso por onde passastes para consumar a nossa redenção no Calvário. Oxalá esta meditação dos mistérios da vossa Paixão nos encha o coração de compunção por nossos pecados e de reconhecimento pelo vosso grande amor para conosco.
\end{flushleft}
\begin{center}
    I.\ Estação \\ Jesus é condenado à morte
\end{center}
\begin{flushleft}
    \VbarRed{} Nós Vos adoramos, ó Jesus e Vos bendizemos. \\
    \RbarRed{} Porque pela vossa Santa Cruz redimistes o mundo.
    \vspace{.2cm} \\
    Do Evangelho segundo São Mateus - (\textcolor{VioletRed2}{Mt 27, 22-23}).
    \vspace{.2cm} \\
    Pilatos perguntou: ``Que farei, então, com Jesus, que é chamado o Cristo?'' Todos responderam: ``Seja crucificado!'' Pilatos falou: ``Mas, que mal ele fez?'' Eles, porém, gritaram com mais força: ``Seja crucificado!''
    \vspace{.2cm} \\
    Pai Nosso, Ave Maria, Glória.
    \vspace{.2cm} \\
    \VbarRed{} Senhor, pequei. \\
    \RbarRed{} Tende piedade e misericórdia de mim.
\end{flushleft}
\begin{center}
    II.\ Estação \\ Jesus toma a sua Cruz
\end{center}
\begin{flushleft}
    \VbarRed{} Nós Vos adoramos, ó Jesus e Vos bendizemos. \\
    \RbarRed{} Porque pela vossa Santa Cruz redimistes o mundo.
    \vspace{.2cm} \\
    Do Evangelho segundo São Mateus - (\textcolor{VioletRed2}{Mt 16, 24-26}).
    \vspace{.2cm} \\
    Então Jesus disse aos discípulos: ``Se alguém quer vir após mim, negue-se a si mesmo, tome a sua cruz e siga-me, pois quem quiser salvar sua vida, a perderá; mas quem perder sua vida por causa de mim, a encontrará. Com efeito, que adianta a alguém ganhar o mundo inteiro, mais arruinar a sua vida? Que poderia dar em troca de sua vida?
    \vspace{.2cm} \\
    Pai Nosso, Ave Maria, Glória.
    \vspace{.2cm} \\
    \VbarRed{} Senhor, pequei. \\
    \RbarRed{} Tende piedade e misericórdia de mim.
\end{flushleft}
\begin{center}
    III.\ Estação \\ Jesus cai sob o peso da Cruz
\end{center}
\begin{flushleft}
    \VbarRed{} Nós Vos adoramos, ó Jesus e Vos bendizemos. \\
    \RbarRed{} Porque pela vossa Santa Cruz redimistes o mundo.
    \vspace{.2cm} \\
    Do Livro do Profeta Isaías - (\textcolor{VioletRed2}{Is 50, 5-7}).
    \vspace{.2cm} \\
    O Senhor Deus abriu-me os ouvidos, e eu não fui rebelde, nem recuei. Apresentei as costas aos que me feriam, e a face aos que me arrancavam a barba; não desviei o rosto dos insultos e dos escarros. O Senhor Deus é quem me ajuda: por isso, não fiquei envergonhado. Permaneço com o rosto impassível, duro como pedra, porque sei que não serei envergonhado.
    \vspace{.2cm} \\
    Pai Nosso, Ave Maria, Glória.
    \vspace{.2cm} \\
    \VbarRed{} Senhor, pequei. \\
    \RbarRed{} Tende piedade e misericórdia de mim.
\end{flushleft}\begin{center}
    IV.\ Estação \\ Jesus encontra Maria, sua Santíssima Mãe
\end{center}
\begin{flushleft}
    \VbarRed{} Nós Vos adoramos, ó Jesus e Vos bendizemos. \\
    \RbarRed{} Porque pela vossa Santa Cruz redimistes o mundo.
    \vspace{.2cm} \\
    Do Evangelho segundo São Lucas - (\textcolor{VioletRed2}{Lc 2, 34-35}).
    \vspace{.2cm} \\
    Simeão os abençoou e disse a Maria, sua Mãe: ``Este é destinado a ser causa de queda e de reerguimento de muitos em Israel, e a ser sinal de contradição. Assim serão revelados os pensamentos de muitos corações.\ Quanto a ti, uma espada te transpassará a alma''.
    \vspace{.2cm} \\
    Pai Nosso, Ave Maria, Glória.
    \vspace{.2cm} \\
    \VbarRed{} Senhor, pequei. \\
    \RbarRed{} Tende piedade e misericórdia de mim.
\end{flushleft}
\newpage
\begin{center}
    V.\ Estação \\ Simão Cirineu ajuda Jesus a carregar a Cruz
\end{center}
\begin{flushleft}
    \VbarRed{} Nós Vos adoramos, ó Jesus e Vos bendizemos. \\
    \RbarRed{} Porque pela vossa Santa Cruz redimistes o mundo.
    \vspace{.2cm} \\
    Do Evangelho segundo São Lucas - (\textcolor{VioletRed2}{Lc 23, 26}).
    \vspace{.2cm} \\
    Enquanto levavam Jesus, agarraram um certo Simão de Cirene, que voltava do campo, e puseram-lhe a cruz aos ombros, para que a carregasse atrás de Jesus.
    \vspace{.2cm} \\
    Pai Nosso, Ave Maria, Glória.
    \vspace{.2cm} \\
    \VbarRed{} Senhor, pequei. \\
    \RbarRed{} Tende piedade e misericórdia de mim.
\end{flushleft}
\begin{center}
    VI.\ Estação \\ Uma piedosa mulher enxuga a face de Jesus
\end{center}
\begin{flushleft}
    \VbarRed{} Nós Vos adoramos, ó Jesus e Vos bendizemos. \\
    \RbarRed{} Porque pela vossa Santa Cruz redimistes o mundo.
    \vspace{.2cm} \\
    Do Livro do Profeta Isaías - (\textcolor{VioletRed2}{Is 53, 1-3}).
    \vspace{.2cm} \\
    Quem acreditou naquilo que ouvimos, a quem foi revelado o braço do Senhor? Cresceu diante dele como um renovo, como raiz que nasce da terra seca: Não tinha aparência nem beleza para que o olhássemos, nem formosura que nos atraísse. Foi desprezado, como o último dos homens, homem de dores, experimentado no sofrimento, e quase escondíamos o rosto diante dele; desprezado, não lhe demos nenhuma importância.
    \vspace{.2cm} \\
    Pai Nosso, Ave Maria, Glória.
    \vspace{.2cm} \\
    \VbarRed{} Senhor, pequei. \\
    \RbarRed{} Tende piedade e misericórdia de mim.
\end{flushleft}
\newpage
\begin{center}
    VII.\ Estação\\ Jesus cai pela segunda vez
\end{center}
\begin{flushleft}
    \VbarRed{} Nós Vos adoramos, ó Jesus e Vos bendizemos. \\
    \RbarRed{} Porque pela vossa Santa Cruz redimistes o mundo.
    \vspace{.2cm} \\
    Do Livro do Profeta Isaías - (\textcolor{VioletRed2}{Is 53, 4-5}).
    \vspace{.2cm} \\
    Entretanto, ele assumiu as nossas fraquezas, e as nossas dores, ele as suportou. E nós achávamos que ele era um castigado, alguém por Deus ferido e humilhado. Mas ele foi ferido por causa de nossas iniquidades, esmagado por causa de nossos crimes. O castigo que nos dá a paz caiu sobre ele, por seus ferimentos fomos curados.
    \vspace{.2cm} \\
    Pai Nosso, Ave Maria, Glória.
    \vspace{.2cm} \\
    \VbarRed{} Senhor, pequei. \\
    \RbarRed{} Tende piedade e misericórdia de mim.
\end{flushleft}
\begin{center}
    VIII.\ Estação \\ Jesus consola as mulheres de Jerusalém
\end{center}
\begin{flushleft}
    \VbarRed{} Nós Vos adoramos, ó Jesus e Vos bendizemos. \\
    \RbarRed{} Porque pela vossa Santa Cruz redimistes o mundo.
    \vspace{.2cm} \\
    Do Evangelho segundo São Lucas - (\textcolor{VioletRed2}{Lc 23, 27-28}).
    \vspace{.2cm} \\
    Seguia-o uma grande multidão do povo, bem como mulheres, que batiam no peito e choravam por ele. Jesus, porém, voltou-se para elas e disse: ``Mulheres de Jerusalém, não choreis por mim! Chorai por vós mesmas e por vossos filhos!''
    \vspace{.2cm} \\
    Pai Nosso, Ave Maria, Glória.
    \vspace{.2cm} \\
    \VbarRed{} Senhor, pequei. \\
    \RbarRed{} Tende piedade e misericórdia de mim.
\end{flushleft}
\newpage
\begin{center}
    IX.\ Estação \\ Jesus cai pela terceira vez
\end{center}
\begin{flushleft}
    \VbarRed{} Nós Vos adoramos, ó Jesus e Vos bendizemos. \\
    \RbarRed{} Porque pela vossa Santa Cruz redimistes o mundo.
    \vspace{.2cm} \\
    Do Evangelho segundo São Mateus - (\textcolor{VioletRed2}{Mt 11, 28-30}).
    \vspace{.2cm} \\
    Vinde a mim, todos os que estais cansados e carregados de fardos, e eu vos darei descanso. Tomai sobre vós o meu jugo e aprendei de mim, porque sou manso e humilde de coração, e encontrais descanso para vós, pois o meu julgo é suave e o meu fardo é leve.
    \vspace{.2cm} \\
    Pai Nosso, Ave Maria, Glória.
    \vspace{.2cm} \\
    \VbarRed{} Senhor, pequei. \\
    \RbarRed{} Tende piedade e misericórdia de mim.
\end{flushleft}
\begin{center}
    X.\ Estação \\ Jesus é despojado de suas vestes
\end{center}
\begin{flushleft}
    \VbarRed{} Nós Vos adoramos, ó Jesus e Vos bendizemos. \\
    \RbarRed{} Porque pela vossa Santa Cruz redimistes o mundo.
    \vspace{.2cm} \\
    Do Evangelho segundo São Marcos - (\textcolor{VioletRed2}{Mc 15, 24}).
    \vspace{.2cm} \\
    Eles o crucificaram e repartiram suas vestes, tiram a sorte sobre elas, para ver que parte tocaria a cada um.
    \vspace{.2cm} \\
    Pai Nosso, Ave Maria, Glória.
    \vspace{.2cm} \\
    \VbarRed{} Senhor, pequei. \\
    \RbarRed{} Tende piedade e misericórdia de mim.
\end{flushleft}
\begin{center}
    XI.\ Estação \\ Jesus é pregado na Cruz
\end{center}
\begin{flushleft}
    \VbarRed{} Nós Vos adoramos, ó Jesus e Vos bendizemos. \\
    \RbarRed{} Porque pela vossa Santa Cruz redimistes o mundo.
    \vspace{.2cm} \\
    Do Evangelho segundo São Lucas - (\textcolor{VioletRed2}{Lc 23, 33-34}).
    \vspace{.2cm} \\
    Quando chegaram ao lugar chamado Calvário, ali crucificaram Jesus e os malfeitores: um à sua direita e outro à sua esquerda. Jesus dizia: ``Pai, perdoa-lhes! Eles não sabem o que fazem!'' Então repartiram suas vestes tirando a sorte.
    \vspace{.2cm} \\
    Pai Nosso, Ave Maria, Glória.
    \vspace{.2cm} \\
    \VbarRed{} Senhor, pequei. \\
    \RbarRed{} Tende piedade e misericórdia de mim.
\end{flushleft}
\begin{center}
    XII.\ Estação \\ Jesus morre na Cruz
\end{center}
\begin{flushleft}
    \VbarRed{} Nós Vos adoramos, ó Jesus e Vos bendizemos. \\
    \RbarRed{} Porque pela vossa Santa Cruz redimistes o mundo.
    \vspace{.2cm} \\
    Do Evangelho segundo São João - (\textcolor{VioletRed2}{Jo 19, 28-30}).
    \vspace{.2cm} \\
    Em seguida, sabendo Jesus que tudo estava consumado, para que se cumprisse a Escritura, disse: ``Tenho sede!'' Havia ali uma vasilha cheia de vinagre. Fixaram uma esponja embebida em vinagre num ramo de hissopo e a levaram à sua boca. Depois que tomou o vinagre, ele disse: ``Está consumado''. E, inclinando a cabeça, entregou o espírito.
    \vspace{.2cm} \\
    Pai Nosso, Ave Maria, Glória.
    \vspace{.2cm} \\
    \VbarRed{} Senhor, pequei. \\
    \RbarRed{} Tende piedade e misericórdia de mim.
\end{flushleft}
\begin{center}
    XIII.\ Estação \\ Jesus é descido da Cruz e entregue à sua Mãe
\end{center}
\begin{flushleft}
    \VbarRed{} Nós Vos adoramos, ó Jesus e Vos bendizemos. \\
    \RbarRed{} Porque pela vossa Santa Cruz redimistes o mundo.
    \vspace{.2cm} \\
    Do Evangelho segundo São João - (\textcolor{VioletRed2}{Jo 19, 38-40}).
    \vspace{.2cm} \\
    Depois disso, José de Arimateia, que era discípulo de Jesus, porém às escondidas por medo dos judeus, pediu a Pilatos permissão para retirar o corpo de Jesus. Pilatos o permitiu, e José foi e retirou o corpo. Veio também Nicodemos, aquele que anteriormente tinha ido a Jesus de noite. Ele trouxe uma mistura de mirra e aloés, cerca de cem libras. Eles pegaram o corpo de Jesus e o envolveram, com os perfumes, em faixas de linho, ao modo de como os judeus costumam sepultar.
    \vspace{.2cm} \\
    Pai Nosso, Ave Maria, Glória.
    \vspace{.2cm} \\
    \VbarRed{} Senhor, pequei. \\
    \RbarRed{} Tende piedade e misericórdia de mim.
\end{flushleft}
\newpage
\begin{center}
    XIV.\ Estação \\ Jesus é colocado no sepulcro
\end{center}
\begin{flushleft}
    \VbarRed{} Nós Vos adoramos, ó Jesus e Vos bendizemos. \\
    \RbarRed{} Porque pela vossa Santa Cruz redimistes o mundo.
    \vspace{.2cm} \\
    Do Evangelho segundo São João - (\textcolor{VioletRed2}{Jo 19, 40-42}).
    \vspace{.2cm} \\
    Eles pegaram o corpo de Jesus e o envolveram, com os perfumes, em faixas de linho, ao modo de como os judeus costumam sepultar. No lugar onde Jesus fora crucificado, havia um jardim e, no jardim, um túmulo novo, no qual ninguém ainda havia sido posto. Como era o dia de preparação dos judeus e o túmulo estava perto, ali puseram Jesus.
    \vspace{.2cm} \\
    Pai Nosso, Ave Maria, Glória.
    \vspace{.2cm} \\
    \VbarRed{} Senhor, pequei. \\
    \RbarRed{} Tende piedade e misericórdia de mim.
    \vspace{.2cm} \\
    \textit{No fim:}
    \vspace{.2cm} \\
    Jesus, morto por mim, concedei-me a graça de morrer num ato de perfeita caridade para convosco. Santa Maria, Mãe de Deus, rogai por mim, agora e na hora da minha morte. São José, meu Pai e Senhor, alcançai-me que morra com a morte dos justos. Amém.
    \vspace{.2cm} \\
    \textbf{\textit{Oremos:}} Ó Deus, Pai todo-poderoso, que em Jesus Cristo, teu Filho, assumiste as chagas e os sofrimentos da humanidade, hoje tenho a coragem de Te suplicar, como o ladrão arrependido: ``Lembra-Te de mim''. Estou aqui, sozinho na tua presença, na escuridão dessa prisão, pobre, nu, faminto e desprezado, e peço-Te para derramares sobre as minhas feridas o óleo do perdão e da consolação e o vinho duma fraternidade que fortalece o coração. Cura-me com tua graça e ensina-me a manter a esperança no meio do desespero. Continua, Pai misericordioso, a confiar em mim, a dar-me sempre uma nova oportunidade, a abraçar-me no teu amor infinito. Com a tua ajuda e o dom do Espírito Santo, também eu serei capaz de Te reconhecer e servir nos meus irmãos. Amém.
    \vspace{.2cm} \\
    Em nome do Pai, \grecrossRed{} e do Filho, e do Espírito Santo. Amém.
\end{flushleft}
\newpage
\begin{center}
    \textbf{Ofício da Imaculada Conceição}
\end{center}
\begin{center}
    Sinal da Cruz
\end{center}
\begin{flushleft}
    Pelo Sinal, \grecrossRed{} da Santa Cruz, livrai-nos Deus, \grecrossRed{} Nosso Senhor, dos nossos \grecrossRed{} inimigos. Em nome do Pai, \grecrossRed{} e do Filho, e do Espírito Santo. Amém.
    \vspace{.2cm} \\
    Deus vos salve, filha de Deus Pai! \\
    Deus vos salve, mãe de Deus Filho! \\
    Deus vos salve, esposa do Espírito Santo! \\
    Deus vos salve, sacrário da Santíssima Trindade!
\end{flushleft}
\begin{center}
    Matinas
\end{center}
\begin{flushleft}
    Agora, lábios meus dizei e anunciai os grandes louvores da Virgem, Mãe de Deus. Sede em meu favor, Virgem soberana, livrai-nos do inimigo com vosso valor. Glória seja ao Pai, ao Filho e ao Amor também, que é um só Deus em pessoas três, agora e sempre e sem fim. Amém.
\end{flushleft}
\begin{center}
    \textcolor{VioletRed2}{Hino}
\end{center}
\begin{multicols}{2}
    Deus vos salve, \\
    Virgem Senhora do mundo \\
    rainha dos céus, \\
    e das virgens, Virgem.
    \vspace{.2cm} \\
    Estrela da manhã \\
    Deus vos Salve, \\
    cheia de graça divina, \\
    formosa e louçã.
    \vspace{.2cm} \\
    Dai pressa, Senhora, \\
    em favor do mundo, \\
    pois vos reconhece \\
    como defensora.
    \vspace{.2cm} \\
    Deus vos nomeou \\
    desde a eternidade \\
    para a mãe do Verbo \\
    com o qual criou
    \vspace{.2cm} \\
    Terra, mar e céus, \\
    e vos escolheu, \\
    quando Adão pecou, \\
    por esposa de Deus.
    \vspace{.2cm} \\
    Deus a escolheu \\
    e, já muito antes, \\
    em seu tabernáculo \\
    morada lhe deu.
    \vspace{.2cm} \\
    Ouvi, Mãe de Deus, \\
    minha oração. \\
    Toquem em vosso peito \\
    os clamores meus.
\end{multicols}
\newpage
\begin{center}
    \textcolor{VioletRed2}{Oração}
\end{center}
\begin{flushleft}
    Santa Maria, Rainha dos céus, Mãe de Nosso Senhor Jesus Cristo, Senhora do mundo, que a nenhum pecador desamparais e nem desprezais, ponde, Senhora, em mim os olhos de vossa piedade e alcançai de Vosso amado Filho o perdão de todos os meus pecados, para que eu, que agora venero com devoção Vossa Santa e Imaculada Conceição, mereça na outra vida alcançar o prêmio da bem-aventurança, pelo merecimento de Vosso bendito Filho Jesus Cristo, Nosso Senhor, que com o Pai e o Espírito Santo vive e reina para Sempre. Amém.
\end{flushleft}
\begin{center}
    Laudes
\end{center}
\begin{flushleft}
    Sede em meu favor, Virgem soberana, livrai-me do inimigo com o Vosso valor. Glória seja ao Pai, ao Filho e ao Amor também, que é um só Deus em Pessoas três, agora e sempre, e sem fim. Amém.
\end{flushleft}
\begin{center}
    \textcolor{VioletRed2}{Hino}
\end{center}
\begin{multicols}{2}
    Deus Vos salve, mesa \\
    para Deus ornada, \\
    coluna sagrada, \\
    de grande firmeza.
    \vspace{.2cm} \\
    Casa dedicada \\
    a Deus sempiterno. \\
    Sempre preservada \\
    Virgem, do pecado.
    \vspace{.2cm} \\
    Antes que nascida \\
    fostes, Virgem Santa, \\
    no ventre ditoso \\
    de Ana concebida.
    \vspace{.2cm} \\
    Sois Mãe criadora \\
    dos mortais viventes. \\
    Sois dos santos porta, \\
    dos anjos Senhora.
    \vspace{.2cm} \\
    Sois forte esquadrão \\
    contra o inimigo. \\
    Estrela de Jacó, \\
    refúgio do cristão.
    \vspace{.2cm} \\
    A Virgem criou \\
    Deus no Espírito Santo, \\
    e todas as suas obras, \\
    com Ela as ornou.
    \vspace{.2cm} \\
    Ouvi, Mãe de Deus, \\
    minha oração. \\
    Toque em Vosso peito \\
    os clamores meus.
\end{multicols}
\begin{center}
    \textcolor{VioletRed2}{Oração}
\end{center}
\begin{flushleft}
    Santa Maria, Rainha dos céus, Mãe de Nosso Senhor Jesus Cristo, Senhora do mundo, que a nenhum pecador desamparais e nem desprezais, ponde, Senhora, em mim os olhos de Vossa piedade e alcançai-me de Vosso amado Filho o perdão de todos os meus pecados, para que eu, que agora venero com devoção, Vossa Santa e Imaculada Conceição, mereça na outra vida alcançar o prêmio da bem-aventurança, pelo merecimento do Vosso bendito Filho, Jesus Cristo, Nosso Senhor, que com o Pai e o Espírito Santo vive e reina para sempre. Amém.
\end{flushleft}
\begin{center}
    Terça
\end{center}
\begin{flushleft}
    Sede em meu favor, Virgem soberana, livrai-me do inimigo com o Vosso valor. Glória seja ao Pai, ao Filho e ao Amor também, que é um só Deus em Pessoas três, agora e sempre, e sem fim. Amém.
\end{flushleft}
\begin{center}
    \textcolor{VioletRed2}{Hino}
\end{center}
\begin{multicols}{2}
    Deus Vos salve trono \\
    do grão Salomão, \\
    arca do concerto \\
    velo de Gedeão!
    \vspace{.2cm} \\
    Íris do céu clara, \\
    sarça da visão, \\
    favo de Sansão, \\
    fluorescente vara,
    \vspace{.2cm} \\
    A qual escolheu \\
    para ser Mãe sua, \\
    e de Vós nasceu \\
    o Filho de Deus.
    \vspace{.2cm} \\
    Assim Vos livrou \\
    da culpa original \\
    De nenhum pecado \\
    há em Vós sinal.
    \vspace{.2cm} \\
    Vós, que habitais \\
    lá nas alturas, \\
    e tendes Vosso Trono \\
    entre as nuvens puras.
    \vspace{.2cm} \\
    Ouvi, Mãe de Deus, \\
    minha oração. \\
    Toque em Vosso peito \\
    os clamores meus.
\end{multicols}
\begin{center}
    \textcolor{VioletRed2}{Oração}
\end{center}
\begin{flushleft}
    Santa Maria, Rainha dos céus, Mãe de Nosso Senhor Jesus Cristo, Senhora do mundo, que a nenhum pecador desamparais e nem desprezais, ponde, Senhora, em mim os olhos de Vossa piedade e alcançai-me de Vosso amado Filho o perdão de todos os meus pecados, para que eu, que agora venero com devoção, Vossa Santa e Imaculada Conceição, mereça na outra vida alcançar o prêmio da bem-aventurança, pelo merecimento do Vosso bendito Filho, Jesus Cristo, Nosso Senhor, que com o Pai e o Espírito Santo vive e reina para sempre. Amém.
\end{flushleft}
\newpage
\begin{center}
    Sexta
\end{center}
\begin{flushleft}
    Sede em meu favor, Virgem soberana, livrai-me do inimigo com o Vosso valor. Glória seja ao Pai, ao Filho e ao Amor também, que é um só Deus em Pessoas três, agora e sempre, e sem fim. Amém.
\end{flushleft}
\begin{center}
    \textcolor{VioletRed2}{Hino}
\end{center}
\begin{multicols}{2}
    Deus Vos salve Virgem, \\
    da Trindade templo, \\
    alegria dos anjos, \\
    da pureza exemplo.
    \vspace{.2cm} \\
    Que alegrais os tristes \\
    com vossa clemência, \\
    horto de deleite, \\
    palma de paciência.
    \vspace{.2cm} \\
    Sois terra bendita \\
    e sacerdotal. \\
    Sois da castidade, \\
    símbolo real.
    \vspace{.2cm} \\
    Cidade do Altíssimo, \\
    porta oriental, \\
    sois a mesma graça, \\
    Virgem singular.
    \vspace{.2cm} \\
    Qual lírio cheiroso \\
    entre espinhas duras \\
    tal sois Vós, Senhora, \\
    entre as criaturas.
    \vspace{.2cm} \\
    Ouvi, Mãe de Deus, \\
    minha oração. \\
    Toque em Vosso peito \\
    os clamores meus.
\end{multicols}
\begin{center}
    \textcolor{VioletRed2}{Oração}
\end{center}
\begin{flushleft}
    Santa Maria, Rainha dos céus, Mãe de Nosso Senhor Jesus Cristo, Senhora do mundo, que a nenhum pecador desamparais e nem desprezais, ponde, Senhora, em mim os olhos de Vossa piedade e alcançai-me de Vosso amado Filho o perdão de todos os meus pecados, para que eu, que agora venero com devoção, Vossa Santa e Imaculada Conceição, mereça na outra vida alcançar o prêmio da bem-aventurança, pelo merecimento do Vosso bendito Filho, Jesus Cristo, Nosso Senhor, que com o Pai e o Espírito Santo vive e reina para sempre. Amém.
\end{flushleft}
\newpage
\begin{center}
    Nona
\end{center}
\begin{flushleft}
    Sede em meu favor, Virgem soberana, livrai-me do inimigo com o Vosso valor. Glória seja ao Pai, ao Filho e ao Amor também, que é um só Deus em Pessoas três, agora e sempre, e sem fim. Amém.
\end{flushleft}
\begin{center}
    \textcolor{VioletRed2}{Hino}
\end{center}
\begin{multicols}{2}
    Deus Vos salve cidade \\
    de torres guarnecida, \\
    de Davi com armas \\
    bem fortalecida.
    \vspace{.2cm} \\
    De suma caridade \\
    sempre abrasada. \\
    Do dragão à força \\
    foi por Vós prostrada.
    \vspace{.2cm} \\
    Ó mulher tão forte! \\
    Ó invicta Judite! \\
    Que Vós alentastes \\
    o sumo Davi!
    \vspace{.2cm} \\
    Do Egito o curador \\
    de Raquel nasceu, \\
    Do mundo o Salvador, \\
    Maria no-lo deu.
    \vspace{.2cm} \\
    Toda é formosa \\
    minha companheira; \\
    n'Ela não há mácula \\
    da culpa primeira.
    \vspace{.2cm} \\
    Ouvi, Mãe de Deus, \\
    minha oração. \\
    Toque em Vosso peito \\
    os clamores meus.
\end{multicols}
\begin{center}
    \textcolor{VioletRed2}{Oração}
\end{center}
\begin{flushleft}
    Santa Maria, Rainha dos céus, Mãe de Nosso Senhor Jesus Cristo, Senhora do mundo, que a nenhum pecador desamparais e nem desprezais; ponde, Senhora, em mim os olhos de Vossa piedade e alcançai-me de Vosso amado Filho o perdão de todos os meus pecados, para que eu, que agora venero com devoção, Vossa Santa e Imaculada Conceição, mereça na outra vida alcançar o prêmio da bem-aventurança, pelo merecimento do Vosso bendito Filho, Jesus Cristo, Nosso Senhor, que com o Pai e o Espírito Santo vive e reina para sempre. Amém.
\end{flushleft}
\newpage
\begin{center}
    Vésperas
\end{center}
\begin{flushleft}
    Sede em meu favor, Virgem soberana, livrai-me do inimigo com o Vosso valor. Glória seja ao Pai, ao Filho e ao Amor também, que é um só Deus em Pessoas três, agora e sempre, e sem fim. Amém.
\end{flushleft}
\begin{center}
    \textcolor{VioletRed2}{Hino}
\end{center}
\begin{multicols}{2}
    Deus Vos salve, relógio \\
    que, andando atrasado \\
    serviu de sinal \\
    ao Verbo Encarnado.
    \vspace{.2cm} \\
    Para que o homem suba \\
    às sumas alturas, \\
    desce Deus do céu \\
    para as criaturas.
    \vspace{.2cm} \\
    Com os raios claros \\
    do Sol de Justiça \\
    Resplandece a Virgem \\
    dando ao sol cobiça
    \vspace{.2cm} \\
    Sois lírio formoso \\
    que cheiro respira \\
    entre os espinhos \\
    da serpente a ira.
    \vspace{.2cm} \\
    Vós aquebrantais \\
    com Vosso poder. \\
    Os cegos errados \\
    Vós alumiais.
    \vspace{.2cm} \\
    Fizestes nascer \\
    Sol tão fecundo, \\
    e, como com nuvens \\
    cobristes o mundo.
    \vspace{.2cm} \\
    Ouvi, Mãe de Deus, \\
    minha oração. \\
    Toque em Vosso peito \\
    os clamores meus.
\end{multicols}
\begin{center}
    \textcolor{VioletRed2}{Oração}
\end{center}
\begin{flushleft}
    Santa Maria, Rainha dos céus, Mãe de Nosso Senhor Jesus Cristo, Senhora do mundo, que a nenhum pecador desamparais e nem desprezais; ponde, Senhora, em mim os olhos de Vossa piedade e alcançai-me de Vosso amado Filho o perdão de todos os meus pecados, para que eu, que agora venero com devoção, Vossa Santa e Imaculada Conceição, mereça na outra vida alcançar o prêmio da bem-aventurança, pelo merecimento do Vosso bendito Filho, Jesus Cristo, Nosso Senhor, que com o Pai e o Espírito Santo vive e reina para sempre. Amém.
\end{flushleft}
\newpage
\begin{center}
    Completas
\end{center}
\begin{flushleft}
    Rogai a Deus, Vós, Virgem, nos converta, que a Sua ira se aparte de nós. Sede em meu favor, Virgem soberana, livrai-me do inimigo com o Vosso valor. Glória seja ao Pai, ao Filho e ao Amor também, que é um só Deus em Pessoas três, agora e sempre, e sem fim. Amém.
\end{flushleft}
\begin{center}
    \textcolor{VioletRed2}{Hino}
\end{center}
\begin{multicols}{2}
    Deus Vos salve, Virgem, \\
    Mãe Imaculada, \\
    rainha de clemência \\
    de estrela coroada.
    \vspace{.2cm} \\
    Vós sobre os anjos \\
    sois purificada; \\
    de Deus a mão direita \\
    estás de ouro ornada.
    \vspace{.2cm} \\
    Por Vós, Mãe da graça, \\
    mereçamos ver \\
    a Deus nas alturas \\
    com todo prazer.
    \vspace{.2cm} \\
    Pois sois a esperança \\
    dos pobres errantes, \\
    e seguro porto \\
    dos navegantes.
    \vspace{.2cm} \\
    Estrela do mar \\
    e saúde certa, \\
    e porta que estás \\
    para o céu aberta.
    \vspace{.2cm} \\
    É óleo derramado, \\
    Virgem, Vosso nome, \\
    e os servos Vossos \\
    vos hão sempre amados.
    \vspace{.2cm} \\
    Ouvi, Mãe de Deus, \\
    minha oração. \\
    Toque em Vosso peito \\
    os clamores meus.
\end{multicols}
\begin{center}
    \textcolor{VioletRed2}{Oração}
\end{center}
\begin{flushleft}
    Santa Maria, Rainha dos céus, Mãe de Nosso Senhor Jesus Cristo, Senhora do mundo, que a nenhum pecador desamparais e nem desprezais; ponde, Senhora, em mim os olhos de Vossa piedade e alcançai-me de Vosso amado Filho o perdão de todos os meus pecados, para que eu, que agora venero com devoção, Vossa Santa e Imaculada Conceição, mereça na outra vida alcançar o prêmio da bem-aventurança, pelo merecimento do Vosso bendito Filho, Jesus Cristo, Nosso Senhor, que com o Pai e o Espírito Santo vive e reina para sempre. Amém.
\end{flushleft}
\begin{center}
    \textcolor{VioletRed2}{Oferecimento}
\end{center}
\begin{flushleft}
    Humildes oferecemos a Vós, Virgem Pia, estas orações, porque, em nossa guia, vades Vós adiante e na agonia, Vós nos animeis, ó doce Mãe Maria. Amém.
    \vspace{.2cm} \\
    Em nome do Pai, \grecrossRed{} e do Filho, e do Espírito Santo. Amém.
\end{flushleft}
\newpage
\begin{center}
    \textbf{Sacramento da Confissão}
\end{center}
\begin{center}
    Sinal da Cruz
\end{center}
\begin{flushleft}
    Pelo Sinal, \grecrossRed{} da Santa Cruz, livrai-nos Deus, \grecrossRed{} Nosso Senhor, dos nossos \grecrossRed{} inimigos. Em nome do Pai, \grecrossRed{} e do Filho, e do Espírito Santo. Amém.
\end{flushleft}
\begin{center}
    Vinde, Espírito Santo
\end{center}
\begin{flushleft}
    Vinde, Espírito Santo, enchei os corações dos vossos fiéis e acendei neles o fogo do vosso amor. \\
    \VbarRed{} Enviai o vosso Espírito e tudo será criado. \\
    \RbarRed{} E renovareis a face da terra.
    \vspace{.2cm} \\
    \textbf{\textit{Oremos:}} Ó Deus, que instruístes os corações dos vossos fiéis com a luz do Espírito Santo, concedei-nos amar, no mesmo Espírito, o que é reto, e gozar sempre a sua consolação. Por Cristo, Senhor Nosso. Amém.
    \vspace{.2cm} \\
    Meu Deus e Senhor, vou receber agora o santo Sacramento da Penitência. Ajudai-me para isso com o auxílio de vossa graça; porque nada posso sem vós. Enviai-me o Espírito Santo, para que conheça bem o número e a gravidade de meus pecados, devida e sinceramente deles me arrependa e faça um firme propósito de não pecar mais. Assisti-me com a vossa graça, para que confesse sinceramente os meus pecados e não cale nada do que deva dizer. Dai-me força para me emendar verdadeiramente. Amém.
    \vspace{.2cm} \\
    Santa Maria, Mãe de Deus, rogai por mim, pobre pecador, para que faça uma boa confissão e alcance o perdão de todos os meus pecados. Jesus, Maria, José, esclarecei-me, socorrei-me, salvai-me. Amém. Santo Anjo da Guarda e todos os Anjos e Santos de Deus, rogai por mim nesta hora. Amém.
    \vspace{.2cm} \\
    Pai Nosso, Ave Maria, Glória.
\end{flushleft}
\newpage
\begin{center}
    Exame de Consciência \\
    \hfill{} \break{}
    \textcolor{VioletRed2}{Os Dez Mandamentos da Lei de Deus} \\
    \hfill{} \break{}
    1\textordmasculine{} Mandamento
\end{center}
\begin{flushleft}
    Eu sou o Senhor, teu Deus, que te fez sair da terra do Egito, da casa da escravidão. Não terás outros deuses diante de mim. Não farás para ti imagem esculpida de nada que se assemelhe ao que existe lá em cima, nos céus, ou embaixo, na terra, ou nas águas que estão abaixo da terra. Não te prostrarás diante desses deuses e não os servirás. (Amar a Deus sobre todas as coisas). \\
\end{flushleft}
\begin{itemize}
    \item Deixei de rezar a oração da manhã e da noite por negligência? (pecado venial);
    \item Rezei orações sem devoção, rindo ou voluntariamente distraído? (pecado venial);
    \item Neguei alguma verdade da fé? (pecado mortal);
    \item Falei contra a Religião: zombei de coisas santas, p.ex., da Santa Missa? (pecado mortal ou venial, conforme a matéria e o modo);
    \item Mandei consultar espíritas, feiticeiros, benzendouros ou cartomantes? Fiz feitiços ou usei devoções supersticiosas? (pecado mortal, pode ser venial se for sem plena advertência ou conhecimento);
    \item Desconfiei de Deus, murmurando contra Ele, entregando-me ao desânimo nas desgraças e até ao desespero? (pecado venial muitas vezes; mortal se há desespero da misericórdia de Deus ou da salvação);
    \item Pequei e continuei a pecar contando com a misericórdia divina? (pecado mortal);
    \item Esperei não perder a vida, por milagre de Deus, expondo-me ao risco de vida ou a outro grave perigo? (pecado mortal; venial se não houve plena advertência).
\end{itemize}
\newpage
\begin{center}
    2\textordmasculine{} Mandamento
\end{center}
\begin{flushleft}
    Não pronunciarás o nome do Senhor, teu Deus, em vão.
\end{flushleft}
\begin{itemize}
    \item Disse o nome de Deus ou dos santos sem respeito? (pecado venial);
    \item Jurei falso? (se for verdadeiro juramento falso, é pecado mortal);
    \item Jurei à toa? (pecado venial);
    \item Roguei pragas, proferi maldições? (pecado mortal; venial, se for sem plena advertência, ou a matéria leve);
    \item Blasfemei, disse injúrias contra Deus ou os santos? (pecado mortal)
    \item Não cumpri as promessas que eu fiz a Deus e aos santos? (é grave, se a matéria o for e se teve a intenção de ficar gravemente obrigado. Caso contrário, é leve).
\end{itemize}
\begin{center}
    3\textordmasculine{} Mandamento
\end{center}
\begin{flushleft}
    Lembra-te de guardar o Dia do Senhor.
\end{flushleft}
\begin{itemize}
    \item Faltei por minha culpa à Missa toda ou grande parte dela nos domingos e festas de guarda? (pecado mortal; venial se a parte omitida não é das principais e for pequena);
    \item Trabalhei nos domingos e dias santos sem necessidade? - Diga que trabalho foi e quando tempo (pecado mortal, conforme o trabalho e o tempo).
\end{itemize}
\newpage
\begin{center}
    4\textordmasculine{} Mandamento
\end{center}
\begin{flushleft}
    Honrar teu pai e tua mãe, para que se prolonguem os teus dias na terra que o Senhor, teu Deus, te dá.
\end{flushleft}
\begin{itemize}
    \item Desobedeci, contrariei ou aborreci ou faltei com o respeito a meus superiores? (é geralmente leve; pode ser grave se a matéria é grave);
    \item Deixei de socorrê-los nas necessidades? - Diga em que foi (pecado mortal ou venial conforme a matéria e as circunstâncias);
    \item Desejei a meus pais grande (pecado mortal) ou pequeno mal (venial)?
\end{itemize}
\begin{flushleft}
    Examinem os pais a consciência a respeito de seus filhos se não houve descuido da sua educação e instrução religiosa; se lhes deram bons exemplos; se não deixaram faltar-lhes o necessário. \\
    Para aqueles que são patrões, a respeito dos funcionários: se os trataram bem, pagaram o devido salário, deram instruções e tempo para o cumprimento dos deveres religiosos. \\
    Os cidadãos a respeito do amor à Pátria: revolta contra a devida autoridade; pagamento do imposto e defesa do seu território.  \\
    Os eleitores examinem se voltaram ou não, e em quem votaram: se em pessoas competentes e de princípios cristãos, ou em outros só por serem amigos.
\end{flushleft}
\begin{center}
    5\textordmasculine{} Mandamento
\end{center}
\begin{flushleft}
    Não matarás.
\end{flushleft}
\begin{itemize}
    \item Briguei com meus irmãos ou outras pessoas? Injuriei, maltratei-os? (pecado venial; mortal quando há gravidade nestes maus tratos);
    \item Matei? Procurei matar? Desejei matar? (pecado grave) Provoquei aborto? (grave e excomunhão) Provoquei acidentes de trânsito? Fui imprudente ao guiar?;
    \item Feri alguém? (conforme o ferimento é grave ou leve);
    \item Tive raiva e inimizade. Desejo de vingança? (pecado mortal ou venial segundo o tempo longo ou curto e a matéria for leve);
    \item Fiz os outros pecarem por maus exemplos, conselhos, palavras ou obras? (pecado mortal ou venial segundo o pecado de que fomos causa).
\end{itemize}
\newpage
\begin{center}
    6\textordmasculine{} Mandamento
\end{center}
\begin{flushleft}
    Não pecarás contra a castidade.
\end{flushleft}
\begin{center}
    9\textordmasculine{} Mandamento
\end{center}
\begin{flushleft}
    Não desejarás a mulher do próximo. \\
    \hfill{} \break{}
    Todos os pecados de luxúria ou impureza são mortais, exceto se não há advertência e pleno consentimento. Os pecados, porém, de falta de pudor ou imodéstia podem ser graves ou leves: depende isto do maior ou menor perigo de impureza, do escândalo que causa e da intenção da pessoa.
\end{flushleft}
\begin{itemize}
    \item Pensei involuntariamente em coisas desonestas?;
    \item Quis ver, ouvir, falar, ler ou fazer coisas desonestas? (dizer com que classe; homem ou mulher, perante ou não, etc, sem citar nomes);
    \item Olhei com prazer, fale, ouvi coisas desonestas?;
    \item Vi figuras ou li escritos imorais? Decotes exagerados? Faltei à modéstia ao vestir-me ou despir-me ou em outras ocasiões? - Fiz ações desonestas ou deixei que as fizessem em mim? Com parentes, com pessoas do mesmo ou de outro sexo? (dizer se é casado ou não, e se a pessoa com que se fez o é ou não);
    \item Evitei filhos no matrimônio? Tomei remédios ou coisas semelhantes para tal fim?
    \item Ensinei. Provoquei ou ajudei a outros por palavras, ações, etc., a cometerem o pecado desonesto?;
    \item Expus-me a ocasiões próximas de pecado, como são certas pessoas ou companhias, teatros e festas indecentes, lugares suspeitos, livros obscenos?;
    \item Os noivos examinem-se sobre se foram castos com suas noivas. Os jovens, sobre as festas, as danças, filmes indecentes, os livros obscenos.
\end{itemize}
\newpage
\begin{center}
    7\textordmasculine{} Mandamento
\end{center}
\begin{flushleft}
    Não roubarás.
\end{flushleft}
\begin{center}
    10\textordmasculine{} Mandamento
\end{center}
\begin{flushleft}
    Não cobiçarás as coisas alheias. \\
    \hfill{} \break{}
    Pecados graves ou leves: segundo os objetos injustamente tirados ou segundo as circunstâncias.
\end{flushleft}
\begin{itemize}
    \item Furtei? Desejei furtar alguma coisa?;
    \item Aceitei, comprei ou tive coisas roubadas? Fiquei com coisas emprestadas ou achadas sem procurar o dono?;
    \item Não paguei ou demorei em pagar as dívidas?;
    \item Prejudiquei aos outros nos seus bens, entregando-lhes objetos, enganando-os no preço ou na medida do que lhes vendi?;
    \item Ajudei outros a fazerem tais pecados por ordem, conselhos ou silêncio culposo? Gastei no jogo? Não sustentei a família?.
\end{itemize}
\begin{center}
    8\textordmasculine{} Mandamento
\end{center}
\begin{flushleft}
    Não apresentarás um falso testamento contra teu próximo.
\end{flushleft}
\begin{itemize}
    \item Menti? Menti com prejuízo para os outros? (geralmente, pecado venial: mortal, se o prejuízo causado for grave);
    \item Murmurei da vida alheia, critiquei ou revelei faltas do próximo? (pecado mortal se a murmuração ou falta descoberta for grave);
    \item Pensei mal do próximo sem razão? (pecado leve ou grave, conforme o que se pensou);
    \item Caluniei? Provoquei ou favoreci críticas ao próximo? (pecado mortal ou venial conforme a falta inventada ou apontada for grave ou leve);
    \item Causei discórdia por minhas murmurações? (pecado mortal ou venial conforme as discórdias causadas).
\end{itemize}
\newpage
\begin{center}
    \textcolor{VioletRed2}{Os Dez Mandamentos da Lei de Deus} \\
    \hfill{} \break{}
    1\textordmasculine{} Mandamento da Igreja
\end{center}
\begin{flushleft}
    Participar da Missa inteira nos domingos e em outras festas de guarda e abster-se de ocupações de trabalho. \\
    \hfill{} \break{}
    Dias de Preceitos da Igreja:
    \begin{itemize}
        \item  Santa Mãe de Deus, Maria;
        \item Epifania do Senhor;
        \item Ascensão do Senhor;
        \item Corpus Christi;
        \item Santos Apóstolos São Pedro e São Paulo;
        \item Assunção de Nossa Senhora;
        \item Todos os Santos;
        \item Imaculada Conceição;
        \item Natal do Senhor.
    \end{itemize}
\end{flushleft}
\begin{center}
    2\textordmasculine{} Mandamento da Igreja
\end{center}
\begin{flushleft}
    Confessar-se ao menos uma vez por ano.
\end{flushleft}
\begin{center}
    3\textordmasculine{} Mandamento da Igreja
\end{center}
\begin{flushleft}
    Receber o sacramento da Eucaristia ao menos pela Páscoa da Ressurreição.
\end{flushleft}
\begin{center}
    4\textordmasculine{} Mandamento da Igreja
\end{center}
\begin{flushleft}
    Jejuar e abster-se de carne, conforme manda a Santa Mãe Igreja. \\
    \hfill{} \break{}
    Dias de jejum e abstinência obrigatórios: \\
    Quarta-Feira de Cinzas e Sexta-Feira Santa.
\end{flushleft}
\begin{center}
    5\textordmasculine{} Mandamento da Igreja
\end{center}
\begin{flushleft}
    Ajudar a Igreja em suas necessidades.
\end{flushleft}
\newpage
\begin{center}
    Textos para Meditação
\end{center}
\begin{flushleft}
    Evangelho segundo São Mateus - \textcolor{VioletRed2}{Mt 5-7}; \\
    Carta de São Paulo aos Romanos - \textcolor{VioletRed2}{Rm 12-15}; \\
    Primeira Carta de São Paulo aos Coríntios - \textcolor{VioletRed2}{1Cor 12-13}; \\
    Carta de São Paulo aos Gálatas - \textcolor{VioletRed2}{Gl 5}; \\
    Carta de São Paulo aos Efésios - \textcolor{VioletRed2}{Ef 4-5}.
\end{flushleft}
\begin{center}
    Ato de Contrição
\end{center}
\begin{flushleft}
    Senhor meu Jesus Cristo, Deus e homem verdadeiro, Criador e Redentor meu, por serdes Vós quem sois, sumamente bom e digno de ser amado sobre todas as coisas, e porque Vos amo e estimo, pesa-me, também, de ter perdido o Céu e merecido o Inferno; e proponho-me firmemente, ajudado com o auxílio da vossa divina graça, emendar-me e nunca mais Vos tornar a ofender. Espero alcançar o perdão de minhas culpas pela vossa infinita misericórdia. Amém.
\end{flushleft}
\begin{center}
    Oração ao Sagrado Coração de Jesus
\end{center}
\begin{flushleft}
    Ó Deus, que no coração de vosso Filho, ferido por nossos pecados, Vos dignais prodigalizar-nos os infinitos tesouros de Amor, fazei, Vos rogamos, que rendendo-Lhe o preito de nossa devoção e piedade, também cumpramos dignamente para com Ele o dever de reparação. Pelo mesmo Jesus Cristo, Senhor Nosso. Amém.
\end{flushleft}
\begin{center}
    Ato de reparação
\end{center}
\begin{flushleft}
    Dulcíssimo Jesus, cuja infinita caridade para com os homens é por eles tão ingratamente correspondida com esquecimentos, friezas e desprezos, eis-nos aqui prostrados na Vossa presença, para Vos desagravarmos, com especiais homenagens, da insensibilidade tão insensata e das nefandas injúrias com que é de toda parte alvejado o Vosso amorosíssimo coração.
    \vspace{.2cm} \\
    Reconhecendo, porém, com a mais profunda dor, que também nós mais de uma vez cometemos as mesmas indignidades, para nós, em primeiro lugar, imploramos a Vossa misericórdia, prontos a expiar não só as próprias culpas, senão também as daqueles que, errando longe do caminho da salvação, ou se obstinam na sua infidelidade, não Vos querendo como pastor e guia, ou, conculcando as promessas do batismo, sacudiram o suavíssimo jugo da Vossa santa lei.
    \newpage
    De todos estes tão deploráveis crimes, Senhor, queremos nós hoje desagravar-Vos, mais particularmente da licença dos costumes e imodéstia do vestido, de tantos laços de corrupção armados à inocência, da violação dos dias santificados, das execrandas blasfêmias contra Vós e Vossos Santos, dos insultos ao Vosso Vigário e a todo o Vosso clero, do desprezo e das horrendas e sacrílegas profanações do Sacramento do divino amor e, enfim, dos atentados e rebeldias das nações contra os direitos e o Magistério da Vossa Igreja.
    \vspace{.2cm} \\
    Oh! Se pudéssemos lavar com o próprio sangue tantas iniquidades!
    \vspace{.2cm} \\
    Entretanto, para reparar a honra divina ultrajada, Vos oferecemos, juntamente com os merecimentos da Virgem Mãe, de todos os santos e almas piedosas, aquela infinita satisfação, que Vós oferecestes ao eterno Pai sobre a cruz, e que não cessais de renovar todos os dias sobre nossos altares.
    \vspace{.2cm} \\
    Ajudai-nos Senhor, com o auxílio da Vossa graça, para que possamos, como é nosso firme propósito, com a vivência da fé, com a pureza dos costumes, com a fiel observância da lei e caridade evangélicas, reparar todos os pecados cometidos por nós e por nosso próximo, impedir, por todos os meios, novas injúrias de Vossa divina Majestade e atrair ao Vosso serviço o maior número de almas possível.
    \vspace{.2cm} \\
    Recebei, ó benigníssimo Jesus, pelas mãos de Maria santíssima reparadora, a espontânea homenagem deste nosso desagravo, e concedei-nos a grande graça de perseverarmos constantes, até à morte, no fiel cumprimento de nossos deveres e no Vosso santo serviço, para que possamos chegar todos à pátria bem-aventurada, onde Vós com o Pai e o Espírito Santo viveis e reinais por todos os séculos dos séculos. Amém.
\end{flushleft}
\begin{center}
    Jaculatórias
\end{center}
\begin{flushleft}
    \VbarRed{} Jesus, manso e humilde de coração. \\
    \RbarRed{} Fazei o nosso coração semelhante ao vosso. \\
    \VbarRed{} Coração sacratíssimo de Jesus. \\
    \RbarRed{} Tende piedade de nós. \\
    \VbarRed{} Coração sacratíssimo e misericordioso de Jesus. \\
    \RbarRed{} Dai-nos a paz.
\end{flushleft}
\newpage
\begin{center}
    Meditação: As Quatro Considerações \\
    \hfill{} \break{}
    \textcolor{VioletRed2}{Primeira Consideração: O Túmulo}
\end{center}
\begin{flushleft}
    Transporta-te em espírito à beira dum túmulo. Imagina-te uma cova. Que é que lá verias? Um cadáver apodrecido, roído por milhares de bichos, tão feio, espalhando cheiro, tão desagradável, que te custaria, suportar tal presença. Eis aqui o homem, rei da terra, a criatura mais formosa e mais nobre deste mundo; um montão de ossos, uma comida de vermes!
    \vspace{.2cm} \\
    E que foi que o reduziu a estado tão horrível? Foi a morte. E quem veio introduzir a morte neste mundo? O pecado. Adão e Eva comeram o fruto proibido, desobedeceram a Deus e foram condenados à morte; e com eles, todo o gênero humano. Eis a consequência dum só pecado mortal! É o pecado que transforma o corpo humano, obra tão esplêndida e artificiosa da onipotência divina, num monturo de podridão. Foi um único pecado mortal, que, num momento, transformou os anjos mais formosos no estado mais feio e mais abominável que há: em demônios. Que grande mal, pois, deve ser o pecado mortal! Sim, muito mais feio e mais horrível do que um cadáver reduzido a podridão.
    \vspace{.2cm} \\
    Contam que, na Antiguidade, um tirano, para atormentar o seu inimigo do modo mais cruel possível, mandou-o ligar vivo a um cadáver. Que castigo horrível dia e noite ser amarrado a um cadáver apodrecido! Muito mais feia, porém, é a alma manchada com o pecado mortal. Essa alma, feita à imagem de Deus, outrora tão formosa, um templo de Deus, mais bela que o mais lindo jardim de flores, agora uma morada, uma escrava do demônio! E, talvez, já haja muito tempo que vendeste, entregaste tua alma ao demônio, dando entrada em teu coração a este teu inimigo capital, pelo pecado grave. Ah! Como és infeliz agora! Em lugar de paz e de alegria, agora remorsos horríveis.
    \vspace{.2cm} \\
    Não permitais, meu Deus, que mais uma vez entregue minha alma imortal ao demônio, dando a meu corpo um prazer proibido, satisfazendo os desejos da carne, que sempre se revolta contra o meu espírito; desta carne que um dia vai ser reduzida a pó e cinza, a uma comida de bichos. Fazei que vença as minhas más inclinações, principalmente esta \textcolor{VioletRed2}{(intenção)} a fim de que um dia meu corpo ressuscite glorioso, para participar da glória celeste.
\end{flushleft}
\newpage
\begin{center}
    \textcolor{VioletRed2}{Segunda Consideração: O Céu}
\end{center}
\begin{flushleft}
    Ergue teu Espírito ao céu. Imagina a coisa mais bela, mais sublime que há neste mundo. Talvez um mimoso jardim no brilho das mais belas flores; ou uma cidade, segundo a descreve S. João no seu Apocalipse, uma cidade com ruas de ouro puro, com portas de pérolas brilhantes, com muros de pedras preciosas. Tudo isso, comparado ao céu, não é mais nada do que a fraca luz duma pequena lâmpada, em comparação ao sol radioso.
    \vspace{.2cm} \\
    Imagina um homem, a quem foi concedido gozar, talvez uma vida inteira, todas as alegrias e prazeres que desde Adão pôde experimentar um pobre mortal. Todos esses gozos e deleites, comparados à Glória do céu, são como uma gota d'água em comparação ao oceano imenso. E este lugar de delícias era destinado para ti; mas, pelo pecado mortal, perdeste o direito de entrar naquela mansão celeste. Um único pecado mortal e perdido está o céu com suas delícias.
    \vspace{.2cm} \\
    Ó meu Deus, que coisa horrível deve ser o pecado mortal, que nos priva de um bem tão grande e sublime! E tu o soubeste, minha alma; e, apesar disto, cometeste o pecado mortal, renunciaste à tua eterna salvação no céu, talvez por algum dinheiro, que tiraste, talvez por outras tantas injustiças que cometeste contra o próximo, talvez por um prazer ilícito, por um pecado desonesto! Ah, que loucura, vender, perder sua eterna felicidade por um momento de prazeres proibidos; trocar o céu por algumas moedas, um punhado, de bens e riquezas passageiras!
    \vspace{.2cm} \\
    Porventura queres de novo perder o céu, cometendo um pecado mortal? Transporta o teu espírito mais uma vez àquela região celeste e contempla o Senhor do céu sentado no seu trono cercado de majestade tremenda, de glória indizível. Imagina os milhares e milhares de espíritos angélicos, trêmulos, prostrados diante do trono de Deus que, com as faces veladas e cheios de respeito e santa reverência cantam incessantemente o ``Santo, Santo, Santo''. Eis, como o céu e a terra se dobram diante da majestade divina --- e como tudo obedece à santíssima vontade de Deus.
    \newpage
    E tu, criatura tão vil e miserável, te atreveste a negar obediência a este Deus tão santo, tão forte, tão poderoso e, ao mesmo tempo, tão bondoso calcando aos pés a sua lei, transgredindo os seus mandamentos, provocando e desprezando a sua justa ira, afligindo amargamente o seu Coração paterno, que tanto te ama e te encheu de tantos benefícios! Prostra-te de joelhos na santíssima presença de Deus e, do fundo de teu coração, dize-lhe: Ó Deus de santidade e de misericórdia infinita, diante de quem o céu e a terra se inclinam, detesto agora sinceramente todos os pecados, com que na minha maldade ofendi a vossa divina majestade, desprezando a bondade do melhor dos pais. Ah, lançai um olhar de compaixão sobre mim, vosso filho ingrato, pois prometo ser agora e sempre um filho obediente, e não tornar a ofender-vos pelo pecado mortal.
\end{flushleft}
\begin{center}
    \textcolor{VioletRed2}{Terceira Consideração: O Inferno}
\end{center}
\begin{flushleft}
    Desce em espírito ao inferno --- e contempla os tormentos horríveis, que lá sofrem os condenados. --- Queimar um pouco o dedo já causa uma grande dor --- por nada deste mundo o deixarias, por uma hora inteira, no fogo. --- Os condenados, porém, sofrem num fogo muito mais ardente do que o nosso, tormentos horríveis. Imagina o rico avarento sepultado e enterrado no meio das chamas, e nem uma gota d'água lhe é concedida para refrescar a sua língua. E quanto tempo são atormentados os condenados? Talvez um dia, um ano, ou um século Isto não seria nada. Judas lá está no inferno há quase dois mil anos sem um sossego ou descanso e sofrerá mais do que estes dois mil, nunca será um pouco aliviado, jamais virá o fim de tantos tormentos e dores, pois as penas são eternas, nunca terão fim.
    \vspace{.2cm} \\
    Os condenados são sempre atormentados por dois pensamentos: ``nunca'' e ``sempre''. Nunca sair, sempre ficar na companhia das criaturas mais perversas e desgraçadas. Agora, pergunta-te, minha alma, qual a causa de tormentos tão horríveis? É o pecado o pecado mortal. E basta um só pecado mortal para te fazer merecedor de tais castigos da justiça divina. Se o bom Deus, o Deus de misericórdia, que não quer a morte do pecador, mas sim que se converta e viva, se Ele castiga tão severamente o pecado, por acaso não deves estremecer e temer de tornar a ofender pelo pecado mortal este juiz tremendo e severo? Ah, promete agora emendar-te. Ainda há tempo. Sim, meu Deus, antes morrer, que vos ofender pelo pecado mortal. Dai-me força para evitá-lo, principalmente este \textcolor{VioletRed2}{(Aqui diga o pecado a que te vês mais inclinado)}.
\end{flushleft}
\newpage
\begin{center}
    \textcolor{VioletRed2}{Quarta Consideração: O Calvário}
\end{center}
\begin{flushleft}
    Transporta teu espírito ao Calvário e contempla a Jesus Crucificado. As dores que teu Salvador sofre são tão horríveis que te deviam mover à compaixão mesmo se fosse o teu inimigo mortal, que lá padecesse. Mas é teu Salvador! Contempla-o: desde a ponta dos pés até a cabeça não há nem um ponto do seu corpo que não fosse martirizado; todo o corpo ferido, todo o corpo uma chaga. A cabeça é atormentada pela coroa de espinhos agudos; a boca pela sede ardente; as mãos e os pés são transpassados por pregos duros; a alma santíssima é abismada no mais profundo abandono.
    \vspace{.2cm} \\
    O verme, quando pisado, pode torcer-se. Jesus nem sequer pode mover-se na Cruz. E quem é aquele que tão cruelmente é maltratado? Talvez um malfeitor? Não. É o mais santo, o mais inocente é o próprio Filho de Deus. E por que se deixou pregar na Cruz? Por causa de teus pecados, para salvar-te da morte eterna do inferno, para reconciliar-te com Deus: Jesus, o Filho de Deus, morre na Cruz. Para salvar o servo, o Filho é condenado à morte! Ó, que amor! E tu soubeste quantas dores, quantos açoites, quantas gotas de seu sangue preciosíssimo custou a teu Jesus tua alma imortal!
    \vspace{.2cm} \\
    E, apesar disto, calcaste o sangue de Deus aos pés, cometendo o pecado mortal. Sim, para pagar a tua desobediência e o teu orgulho, Jesus carregou a sua Cruz. Para satisfazer por teus pensamentos e desejos pecaminosos foram-lhe enterrados aqueles espinhos pungentíssimos. Aquela sede ardente sofreu Jesus por causa de tantas palavras livres e indecentes ou ofensivas ao próximo; e, para pagar tantas ações ilícitas e pecaminosas, Jesus recebeu aqueles açoites horríveis e se deixou cravar com pregos duros no lenho da cruz.
    \vspace{.2cm} \\
    Minha alma, queres pregar mais uma vez a teu Jesus na cruz? Ajoelha-te diante da imagem de teu Jesus crucificado; pede-lhe perdão por tanta ingratidão e promete nunca mais ofender a teu amável Salvador. Dize-lhe de coração contrito: ``Ah, meu Jesus, por amor de vossas cinco chagas, por amor de vosso sangue preciosíssimo, lavai a minha pobre alma de toda a mancha do pecado. Deixai cair sobre minha alma uma só gota de vosso precioso sangue, tão copiosamente derramado, e minha alma será inteiramente purificada. E poderei chamar-me outra vez vosso filho. Ó doce Jesus, que tanto me amais: Fazei que eu vos ame cada vez mais!''
\end{flushleft}
\newpage
\begin{center}
    \textbf{Santa Missa}
\end{center}
\begin{center}
    Oração para antes da Missa \\ \textcolor{VioletRed2}{\scriptsize{(Oração de Santo Tomás de Aquino)}}
\end{center}
\begin{flushleft}
    Deus eterno e todo-poderoso, eis que me aproximo do sacramento do vosso Filho único, Nosso Senhor Jesus Cristo. Impuro, venho à fonte da misericórdia; cego, à luz da eterna claridade; pobre e indigente, ao Senhor do céu e da terra. Imploro, pois, a abundância da vossa liberalidade, para que Vos digneis curar a minha fraqueza, lavar as minhas manchas, iluminar a minha cegueira, enriquecer a minha pobreza, vestir a minha nudez. Que eu receba o Pão dos Anjos, o Rei dos reis e o Senhor dos senhores, com o respeito e a humildade, a contrição e a devoção, a pureza e a fé, o propósito e a intenção que convêm à salvação da minha alma.
    \vspace{.2cm} \\
    Dai-me que receba não só o sacramento do Corpo e do Sangue do Senhor, mas também o seu efeito e a sua força. Ó Deus de mansidão, fazei-me acolher com tais disposições o Corpo que o vosso Filho único, Nosso Senhor Jesus Cristo, recebeu da Virgem Maria, que seja incorporado ao seu Corpo Místico e contado entre os seus membros. Ó Pai cheio de amor, fazei que, recebendo agora o vosso Filho sob o véu do sacramento, possa na eternidade contemplá-la face a face. Vós, que viveis e reinais na unidade do Espírito Santo, por todos os séculos dos séculos. Amém.
\end{flushleft}
\begin{center}
    Comunhão Espiritual \\ \textcolor{VioletRed2}{\scriptsize{(Oração de Santo Afonso Maria de Ligório)}}
\end{center}
\begin{flushleft}
    Meu Jesus, eu creio que estais presente no Santíssimo Sacramento. Amo-Vos sobre todas as coisas e minha alma suspira por Vós, mas, como não posso receber-Vos agora, no Santíssimo Sacramento, vinde ao menos espiritualmente a meu coração. Abraço-me convosco como se já estivésseis comigo; uno-me convosco inteiramente. Ah! Não permitais que eu algum dia me separe de Vós. Ó Jesus, sumo bem e doce amor meu, inflamai meu coração, a fim de que esteja para sempre abrasado em vosso Amor. Amém.
\end{flushleft}
\newpage
\begin{center}
    Anima Christi \\ \textcolor{VioletRed2}{\scriptsize{(Oração de Santo Inácio de Loyola)}}
\end{center}
\begin{flushleft}
    Alma de Cristo, Santificai-me. \\
    Corpo de Cristo, salvai-me. \\
    Sangue de Cristo, inebriai-me. \\
    Água do lado de Cristo, lavai-me. \\
    Paixão de Cristo, confortai-me. \\
    Ó bom Jesus, ouvi-me. \\
    Dentro das Vossa chagas, escondei-me. \\
    Não permitais que eu me separe de Vós. \\
    Do espírito maligno defendei-me. \\
    Na hora da minha morte, chamai-me. \\
    Mandai-me ir para Vós. \\
    Para que Vos louve com os Vossos Santos \\
    Pelos séculos dos séculos. Amém.
\end{flushleft}
\begin{center}
    Orações para depois da Missa \\ \textcolor{VioletRed2}{\scriptsize{(Oração de Santo Tomás de Aquino)}}
\end{center}
\begin{flushleft}
    Dou-vos graças, Senhor santo, Pai onipotente, Deus eterno, a vós que, sem merecimento nenhum de minha parte, mas por efeito de vossa misericórdia, vos dignastes saciar-me, sendo eu pecador e vosso indigno servo, com o corpo adorável e com o sangue precioso do vosso Filho, Nosso Senhor Jesus Cristo.
    \vspace{.2cm} \\
    Eu vos peço que esta comunhão não me seja imputada como uma falta digna de castigo, mas interceda eficazmente para alcançar o meu perdão; seja a armadura da minha fé e o escudo da minha boa vontade; livre-me de meus vícios; apague os meus maus desejos; mortifique a minha concupiscência; aumente em mim a caridade e a paciência, a humildade, a obediência e todas as virtudes; sirva-me de firme defesa contra os embustes de todos os meus inimigos, tanto visíveis como invisíveis; serene e regule perfeitamente todos os movimentos, tanto de minha carne como de meu espírito; una-me firmemente a vós, que sois o único e verdadeiro Deus; e seja enfim a feliz consumação de meu destino.
    \vspace{.2cm} \\
    Dignai-vos, Senhor, eu vos suplico, conduzir-me, a mim pecador, a esse inefável festim onde, com o vosso Filho e o Espírito Santo, sois para os vossos santos luz verdadeira, gozo pleno e alegria eterna, cúmulo de delícias e felicidade perfeita. Pelo mesmo Jesus Cristo, Senhor Nosso. Amém.
\end{flushleft}
\newpage
\begin{center}
    Ave Maria \\ \textcolor{VioletRed2}{\scriptsize{(3 Vezes)}}
\end{center}
\begin{flushleft}
    Ave, Maria, cheia de graça, o Senhor é convosco, bendita sois Vós entre as mulheres e bendito é o fruto do vosso ventre, Jesus. Santa Maria, Mãe de Deus, rogai por nós, pecadores, agora e na hora da nossa morte. Amém.
\end{flushleft}
\begin{center}
    Salve Rainha
\end{center}
\begin{flushleft}
    Salve Rainha, Mãe de misericórdia, vida, doçura, esperança nossa, salve! A vós bradamos, os degredados filhos de Eva. A Vós suspiramos, gemendo e chorando neste vale de lágrimas. Eia, pois advogada nossa, esses vossos olhos misericordiosos a nós volvei, e depois deste desterro mostrai-nos Jesus, bendito fruto do vosso ventre. Ó clemente, ó piedosa, ó doce sempre Virgem Maria.
    \vspace{.2cm} \\
    \VbarRed{} Rogai por nós, Santa Mãe de Deus. \\
    \RbarRed{} Para que sejamos, dignos das promessas de Cristo.
    \vspace{.2cm} \\
    \textbf{\textit{Oremos:}} Deus, nosso refúgio e fortaleza, olhai propício para o povo que a Vós clama; e, pela intercessão da gloriosa e imaculada Virgem Maria, Mãe de Deus, e do Bem-aventurado São José, seu Esposo, e dos vossos bem-aventurados Apóstolos São Pedro e São Paulo e de todos os Santos, ouvi misericordioso e benigno as preces que Vos dirigimos para a conversão dos pecadores, para a liberdade e exaltação da Santa Madre Igreja. Pelo mesmo Cristo Senhor nosso.
    \vspace{.2cm} \\
    \RbarRed{} Amém.
    \vspace{.2cm} \\
    São Miguel Arcanjo, defendei-nos no combate, cobri-nos com o vosso escudo contra os embustes e ciladas do demônio. Subjugue-o Deus, instantemente o pedimos. E vós, príncipe da milícia celeste, pelo divino poder, precipitai no inferno a Satanás e a todos os espíritos malignos que andam pelo mundo para perder as almas.
    \vspace{.2cm} \\
    \RbarRed{} Amém.
    \vspace{.2cm} \\
    \VbarRed{} Sacratíssimo Coração de Jesus. \\
    \RbarRed{} Tende piedade de nós. \\
    \VbarRed{} Sacratíssimo Coração de Jesus. \\
    \RbarRed{} Tende piedade de nós. \\
    \VbarRed{} Sacratíssimo Coração de Jesus. \\
    \RbarRed{} Tende piedade de nós.
\end{flushleft}
\newpage
\begin{center}
    \textbf{Orações Diversas}
\end{center}
\begin{center}
    Oração antes de começar a estudar \\ \textcolor{VioletRed2}{\scriptsize{(Oração de Santo Tomás de Aquino)}}
\end{center}
\begin{flushleft}
    Infalível Criador, que dos tesouros da vossa sabedoria, tiraste as hierarquias doa Anjos colocando-as com ordem admirável no céu e distribuístes o universo com encantável harmonia, Vós que sois a verdadeira fonte de luz e o princípio supremo da sabedoria, difundi sobre as trevas da minha mente o raio do esplendor, removendo as duplas trevas nas quais eu nasci: o pecado e a ignorância. Vós que tornaste fecunda a língua das crianças, tornai erudita a minha língua e espalhai sobre os meus lábios a vossa bênção. Concedei-me o discernimento para entender, a capacidade de reter, a sutileza revelar, a facilidade de aprender, a graça abundante de falar e de escrever. Ensinai-me a começar, regei-me a continuar e perseverar até o término. Vós que sois verdadeiro Deus e verdadeiro homem, e que viveis e reinais pelos séculos dos séculos. Amém.
\end{flushleft}
\begin{center}
    Oração pela aceitação da morte
\end{center}
\begin{flushleft}
    Meu Deus e meu Pai, Senhor da vida e da morte, que para justo castigo das nossas culpas, com um decreto imutável determinastes que todos os homens haviam de morrer, olhai para mim prostrado diante de Vós. Detesto de todo o coração as minhas culpas passadas, pelas quais mereci mil vezes a morte, que aceito agora como o fim de expiá-las e para obedecer à vossa amável vontade. De bom grado morrerei, Senhor, no momento, no lugar e do modo que Vós quiserdes, e aproveitarei até esse instante os dias que me restem de vida para lutar contra os meus defeitos e aumentar o meu amor por Vós, para quebrar os laços que atam o meu coração às criaturas e preparar a minha alma para comparecer à vossa presença; e desde agora me abandono sem reservas nos braços da vossa paternal Providência.
\end{flushleft}
\newpage
\begin{center}
    Oração a São José para santificar o trabalho
\end{center}
\begin{flushleft}
    Ó glorioso São José, modelo de todos os que se consagram ao trabalho, alcançai-me a graça de trabalhar com espírito de penitência, em expiação dos meus pecados; de trabalhar com consciência, pondo o cumprimento do meu dever acima das minha inclinações naturais; de trabalhar com agradecimento e alegria, olhando como uma honra o poder desenvolver por meio do trabalho os dons recebidos de Deus. Alcançai-me a graça de trabalhar com ordem, constância, intensidade e presença de Deus, sem jamais retroceder ante as dificuldades; de trabalhar, acima de tudo, com pureza de intenção e desapego de mim mesmo, tendo sempre diante dos olhos todas as almas e as contas que prestarei a Deus: a do tempo perdido, das habilidades inutilizadas, do bem omitido e das vaidades estéreis em meus trabalhos, tão contrários à obra de Deus. Tudo por Jesus, tudo por Maria, tudo à vossa imitação, ó patriarca São José. Esse será o meu lema na vida e na hora da morte. Amém.
\end{flushleft}
\begin{center}
    Oração para alcançar a virtude da pureza
\end{center}
\begin{flushleft}
    Glorioso São José, pai e protetor das virgens, guarda fiel a quem Deus confiou Jesus Cristo, a perfeita inocência, e Maria, a Virgem das virgens. Eu vos peço por Jesus e Maria, esse duplo tesouro a vós tão caro. Com vosso auxílio, dai-me conservar meu corpo isento de toda mancha, e que puro e casto sirva perpetuamente a Jesus e Maria em perfeita castidade. Amém.
\end{flushleft}
\begin{center}
    Oração para obter vitória contra as tentações
\end{center}
\begin{flushleft}
    Meu Deus, \textit{não me lanceis da vossa presença} (\textcolor{VioletRed2}{Sl 50,13}). Sei perfeitamente que não me abandonareis nunca, se for eu o primeiro a vos abandonar, ai! o que me faz temer esta desgraça é a experiência que tenho na minha fraqueza. Senhor, a vós pertence dar-me a força que hei mister contra o Inferno, que pretende reduzir-me ainda sob a sua escravidão; pelo amor de Jesus Cristo vo-la peço. Ó meu Salvador, estabelecei comigo uma paz perpétua, uma união eternamente indissolúvel. A este efeito, dai-me o vosso santo amor.\ \textit{Aquele que não vos ama é morto} (\textcolor{VioletRed2}{1Jo 3, 14}); a vós toca livrar-me desta desgraçada morte, ó Deus da minha alma! Ah! Pela amarga morte que por mim sofrestes, não permitais, a vós suplico, meu Jesus, consinta eu em perder a vossa amizade. Amo-vos sobre todas as coisas; espero permanecer sempre nos laços do vosso santo amor; neles morrer um dia, e neles viver eternamente. Ó Maria, sois a Mãe e dispensadora da perseverança, de vós é portanto que exijo e espero este grande dom. Amém.
\end{flushleft}
\newpage
\begin{center}
    Oração pelos defuntos \\
    \hfill{} \break{}
    \textcolor{VioletRed2}{Responso}
\end{center}
\begin{flushleft}
    Eu sou a ressurreição e a vida; quem crê em Mim, mesmo que esteja morto, viverá; e quem vive e crê em Mim não morrerá eternamente (\textcolor{VioletRed2}{Jo 11, 25}).
    \vspace{.2cm} \\
    Santos de Deus, vinde em seu auxílio; anjos do Senhor, correi ao seu encontro! Acolhei a(s) sua(s) alma(s), levando-a(s) à presença do Altíssimo.
    \vspace{.2cm} \\
    \VbarRed{} Cristo te(vos) chamou. Ele te(vos) receba, e os anjos te(vos) acompanhem ao seio de Abraão. \\
    \RbarRed{} Acolhei a(s) sua(s) alma(s), levando-a(s) à presença do Altíssimo.
    \vspace{.2cm} \\
    \VbarRed{} Dai-lhe(s), Senhor, o repouso eterno e brilhe para ele(s) a vossa luz. \\
    \RbarRed{} Acolhei a(s) sua(s) alma(s), levando-a(s) à presença do Altíssimo.
    \vspace{.2cm} \\
    \VbarRed{} Senhor, tende piedade de nós. \\
    \RbarRed{} Cristo, tende piedade de nós. \\
    \VbarRed{} Senhor, tende piedade de nós.
    \vspace{.2cm} \\
    Pai nosso.
    \vspace{.2cm} \\
    \VbarRed{} Descanse(m) em paz. \\
    \RbarRed{} Amém.
    \vspace{.2cm} \\
    Oração
    \vspace{.2cm} \\
    Ouvi ó Pai, as nossas preces; sede misericordioso para com o(s) vosso(s) servo(s) \textcolor{VioletRed2}{N.}, que chamastes deste mundo. Concedei-lhe (s) a luz e a paz no convívio dos vossos santos. Por Nosso Senhor Jesus Cristo, na unidade do Espírito Santo. \\
    \RbarRed{} Amém.
    \vspace{.2cm} \\
    Oração
    \vspace{.2cm} \\
    Absolvei, Senhor, a(s) alma(s) do(s) vosso(s) servo(s) \textcolor{VioletRed2}{N.} de todos os laços do pecado, a fim de que, na ressurreição gloriosa, entre os vossos Santos e eleitos, possa(m) ele(s), ressuscitado(s) em seu(s) corpo(s), de novo respirar(em). Por Cristo Nosso Senhor. \\
    \RbarRed{} Amém.
    \vspace{.2cm} \\
    \VbarRed{} Eu sou a ressurreição e a vida; quem crê em Mim, mesmo que esteja morto, viverá; e quem vive e crê em Mim não morrerá eternamente. Dai-lhe(s), Senhor, o repouso eterno. \\
    \RbarRed{} E brilhe(m) para ele(s) a vossa luz.
    \vspace{.2cm} \\
    \VbarRed{} Descanse(m) em paz. \\
    \RbarRed{} Amém.
    \vspace{.2cm} \\
    \VbarRed{} A(s) sua(s) alma(s) e as almas de todos os fiéis defuntos, pela misericórdia de Deus, descansem em paz. \\
    \RbarRed{} Amém.
\end{flushleft}
\begin{center}
    Oração de Fátima
\end{center}
\begin{flushleft}
    Meu Deus, eu creio, adoro, espero e amo-Vos. Peço-Vos perdão para os que não creem, não adoram, não esperam e não Vos amam. Santíssima Trindade, Pai, Filho, Espírito Santo, adoro-Vos profundamente e ofereço-Vos o preciosíssimo Corpo, Sangue, Alma e Divindade de Jesus Cristo, presente em todos os sacrários da terra, em reparação dos ultrajes, sacrilégios e indiferenças com que Ele mesmo é ofendido. E pelos méritos infinitos do Seu Santíssimo Coração e do Coração Imaculado de Maria, peço-Vos a conversão dos pobres pecadores.
\end{flushleft}
\begin{center}
    Veni Creator \\ \textcolor{VioletRed2}{\scriptsize{(Oração para o dia 1 de Janeiro)}}
\end{center}
\begin{flushleft}
    Vinde, Espírito Criador, visitai as almas dos vossos fiéis; enchei de graça celestial os corações que Vós criastes.
    \vspace{.2cm} \\
    Vós, chamado o Consolador, dom do Deus Altíssimo, fonte viva, fogo, caridade, a unção espiritual.
    \vspace{.2cm} \\
    Vós, com Vossos sete dons, sois força da destra de Deus; Vós, o prometido pelo Pai, ditai-nos os gemidos da oração.
    \vspace{.2cm} \\
    Acendei a vossa luz em nossas almas, infundi o vosso amor em nossos corações; e a fraqueza da nossa carne, fortalecei-a com perpétua força.
    \vspace{.2cm} \\
    O inimigo, afugentai-o para longe; dai-nos quanto antes a paz; tendo-Vos por guia e condutor, venceremos todos os perigos.
    \vspace{.2cm} \\
    Por Vós conhecemos o Pai, e também o Filho; e que em Vós, Espírito de ambos, acreditamos em todo tempo.
    \vspace{.2cm} \\
    Glória a Deus Pai, ao Filho que ressuscitou e ao Espírito Santo Consolador. Pelos séculos dos séculos. Amém.
    \vspace{.2cm} \\
    \VbarRed{} Enviai o vosso Espírito e tudo será criado. \\
    \RbarRed{} E renovareis a face da terra.
    \vspace{.2cm} \\
    \textbf{\textit{Oremos:}} Ó Deus, que instruístes os corações dos vossos fiéis com a luz do Espírito Santo, concedei-nos amar, no mesmo Espírito, o que é reto, e gozar sempre a sua consolação. Por Cristo, Senhor Nosso. Amém.
\end{flushleft}
\newpage
\begin{center}
    Te Deum \\ \textcolor{VioletRed2}{\scriptsize{(Oração para o dia 31 de Dezembro)}}
\end{center}
\begin{flushleft}
    A Vós, ó Deus, louvamos; a Vós, Senhor, bendizemos \\
    A Vós, ó eterno Pai, adora toda a terra. \\
    A Vós, todos os Anjos, os Céus e todas as Potestades. \\
    A Vós, os Querubins e Serafins proclamam com incessante vozes: \\
    Santo, Santo, Santo, sois Vós, Senhor, Deus dos exércitos! \\
    Cheios estão os céus e a terra da majestade da vossa glória. \\
    A Vós, o glorioso coro dos Apóstolos. \\
    A Vós, o louvável número dos Profetas. \\
    A Vós, Vos louva o brilhante exército dos Mártires. \\
    A Vós confessa a Santa Igreja por toda a redondeza da terra. \\
    Pai de imensa majestade; \\
    Ao vosso adorável Filho, verdadeiro e único; \\
    E também ao Espírito Santo Consolador. \\
    Vós, ó Cristo, sois o Rei da glória. \\
    Vós sois o Filho eterno do Pai. \\
    Vós, para libertar o homem cuja carne havíeis de tomar, não rejeitastes o seio da Virgem. \\
    Vós, vencido o aguilhão da morte, abriste aos fiéis o Reino dos céus. \\
    Vós estais sentado à mão direita de Deus, na glória do Pai. \\
    Cremos que haveis de vir como Juiz. \\
    Por isso Vos rogamos: socorrei os vossos servos, que remistes com o vosso precioso Sangue. \\
    Permiti que sejamos do número dos vossos Santos na glória eterna. \\
    Salvai, Senhor, o vosso povo, e abençoai a vossa herança. \\
    Governai-os e exaltai-os eternamente. \\
    Todos os dias Vos bendizemos. \\
    E louvamos sempre o vosso Nome, por todos os séculos dos séculos. \\
    Dignai-Vos, Senhor, preservar-nos neste dia de todo o pecado. \\
    Tende piedade de nós, Senhor; tende piedade de nós. \\
    Faça-se, Senhor, a vossa misericórdia sobre nós, conforme esperamos em Vós. \\
    Em Vós, Senhor, esperei; não serei confundido eternamente.
    \vspace{.2cm} \\
    \VbarRed{} Bendito sois, Senhor, Deus de nossos pais. \\
    \RbarRed{} E dignos de louvor e glória pelos séculos.
    \vspace{.2cm} \\
    \VbarRed{} Bendigamos o Pai, o Filho e o Espírito Santo. \\
    \RbarRed{} Louvemo-lo e exaltemo-lo para sempre.
    \vspace{.2cm} \\
    \VbarRed{} Senhor, Vós sois bendito no firmamento dos céus. \\
    \RbarRed{} Sois dignos de louvor e glória para sempre.
    \newpage
    \VbarRed{} Bendiga minha alma ao Senhor. \\
    \RbarRed{} E nunca esqueça os seus muitos benefícios.
    \vspace{.2cm} \\
    \VbarRed{} Ouvi, Senhor, a minha oração. \\
    \RbarRed{} E chegue a Vós o meu clamor.
\end{flushleft}
\begin{center}
    Novena de Santa Teresa do Menino Jesus e da Sagrada Face
\end{center}
\begin{center}
    Sinal da Cruz
\end{center}
\begin{flushleft}
    Pelo Sinal, \grecrossRed{} da Santa Cruz, livrai-nos Deus, \grecrossRed{} Nosso Senhor, dos nossos \grecrossRed{} inimigos. Em nome do Pai, \grecrossRed{} e do Filho, e do Espírito Santo. Amém.
\end{flushleft}
\begin{center}
    Oração
\end{center}
\begin{flushleft}
    Santíssima Trindade: Pai, Filho e Espírito Santo: eu vos agradeço por todas as graças com que enriqueceste a vida de vossa serva, Santa Teresinha do Menino Jesus e da Sagrada Face, nestes 24 anos que passou na terra. E pelos méritos de tão querida santinha, concedei-me a graça que ardentemente vos peço: (\textcolor{VioletRed2}{fazer o pedido}), se for conforme a Vossa Santíssima Vontade e para a salvação de minha alma. Ajudai minha fé e minha esperança, Santa Teresinha, cumprindo mais uma vez vossa promessa de que ficareis no Céu a fazer o bem na terra, permitindo que eu ganhe um rosa em sinal de que alcançarei a graça pedida.
    \vspace{.2cm} \\
    Rezar 24 vezes, por cada ano de Santa Teresinha na terra:
\end{flushleft}
\begin{center}
    Glória \\ \textcolor{VioletRed2}{\scriptsize{(24 vezes)}}
\end{center}
\begin{flushleft}
    Glória ao Pai, e ao Filho e ao Espírito Santo. Assim como era no princípio, agora e sempre, por todos os séculos dos séculos. Amém.
    \vspace{.2cm} \\
    \VbarRed{} Santa Teresinha do Menino Jesus e da Sagrada Face. \\
    \RbarRed{} Rogai por mim.
\end{flushleft}
\newpage
\begin{center}
    Devoções dos Dias da Semana
\end{center}
\begin{itemize}
    \item Domingo: O Dia do Senhor
    \item Segunda-Feira: Almas do Purgatório
    \item Terça-Feira: Anjos da Guarda
    \item Quarta-Feira: São José
    \item Quinta-Feira: Eucaristia
    \item Sexta-feira: Paixão de Cristo
    \item Sábado: Nossa Senhora
\end{itemize}
\begin{center}
    Devoções dos Meses do Ano
\end{center}
\begin{itemize}
    \item Janeiro: Santíssimo Nome de Jesus
    \item Fevereiro: Sagrada Família
    \item Março: São José
    \item Abril: Eucaristia e Espírito Santo
    \item Maio: Virgem Maria
    \item Junho: Sagrado Coração de Jesus
    \item Julho: Preciosíssimo Sangue de Cristo
    \item Agosto: Vocações
    \item Setembro: Bíblia
    \item Outubro: Rosário
    \item Novembro: Fiéis Defuntos
    \item Dezembro: Natal
\end{itemize}
\newpage
\begin{center}
    \textbf{Sumário da Doutrina}
\end{center}
\begin{center}
    Os artigos de Fé
\end{center}
\begin{itemize}
    \item Creio em Deus Pai todo-poderoso.
    \item E em Jesus Cristo, seu Filho único, Nosso Senhor.
    \item Jesus Cristo foi concebido pelo poder do Espírito Santo, nasceu da Virgem Maria.
    \item Jesus Cristo padeceu sob Pôncio Pilatos, foi crucificado, morto e sepultado.
    \item Jesus Cristo desceu aos Infernos, ressuscitou dos mortos no terceiro dia.
    \item Jesus subiu aos céus, está sentado à direita de Deus Pai todo-poderoso.
    \item Donde virá julgar os vivos e os mortos.
    \item Creio no Espírito Santo.
    \item Creio na Igreja Católica.
    \item Creio no perdão dos Pecados.
    \item Creio na ressurreição da carne.
    \item Creio na Vida eterna.
\end{itemize}
\begin{center}
    Os Dez Mandamentos
\end{center}
\begin{enumerate}
    \item Eu sou o Senhor, teu Deus, que te fez sair da terra do Egito, da casa da escravidão. Não terás outros deuses diante de mim. Não farás para ti imagem esculpida de nada que se assemelhe ao que existe lá em cima, nos céus, ou embaixo, na terra, ou nas águas que estão abaixo da terra. Não te prostrarás diante desses deuses e não os servirás.
    \item Não pronunciarás o nome do Senhor, teu Deus, em vão.
    \item Lembra-te de guardar o Dia do Senhor.
    \item Honrar teu pai e tua mãe, para que se prolonguem os teus dias na terra que o Senhor, teu Deus, te dá.
    \item Não matarás.
    \item Não pecarás contra a castidade.
    \item Não roubarás.
    \item Não apresentarás um falso testamento contra teu próximo.
    \item Não desejarás a mulher do próximo.
    \item Não cobiçarás as coisas alheias.
\end{enumerate}
\begin{center}
    Os Cinco Mandamentos da Igreja
\end{center}
\begin{itemize}
    \item Participar da Missa inteira nos domingos e em outras festas de guarda e abster-se de ocupações de trabalho.
    \item Confessar-se ao menos uma vez por ano.
    \item Receber o sacramento da Eucaristia ao menos pela Páscoa da Ressurreição.
    \item Jejuar e abster-se de carne, conforme manda a Santa Mãe Igreja.
    \item Ajudar a Igreja em suas necessidades.
\end{itemize}
\begin{center}
    Lei do jejum e abstinência
\end{center}
\begin{itemize}
    \item Toda Sexta-Feira do ano é dia de penitência, a não ser que coincida com alguma solenidade do calendário litúrgico. Nesse dia os fiéis devem abster-se de comer carne ou outro alimento, ou praticar alguma forma de penitência, principalmente alguma obra de caridade ou algum exercício de piedade.
    \item A Quarta-Feira de Cinzas e a Sexta-Feira Santa, memória da Paixão e Morte de Cristo, são dias de jejum e abstinência. A abstinência pode ser substituída pelos próprios fiéis por outra prática de penitência, caridade ou piedade, particularmente pela participação nesses dias na Sagrada Liturgia.
    \item Idade de obrigação: a abstinência obriga a partir dos 14 anos completos; o jejum a partir dos 18 anos completos até os 60 anos começados.
\end{itemize}
\newpage
\begin{center}
    Os Doze Apóstolos
\end{center}
\begin{enumerate}
    \item Pedro (Simão);
    \item Bartolomeu;
    \item André;
    \item Filipe;
    \item Tomé;
    \item Tiago;
    \item João;
    \item Tiago (O menor);
    \item Judas Tadeu;
    \item Judas Iscariotes - Matias;
    \item Simão (O Zelotes);
    \item Mateus.
\end{enumerate}
\begin{center}
    Os Sete Sacramentos
\end{center}
\begin{enumerate}
    \item Batismo;
    \item Confirmação ou Crisma;
    \item Penitência;
    \item Eucaristia;
    \item Unção dos Enfermos;
    \item Ordem;
    \item Matrimônio.
\end{enumerate}
\begin{center}
    As Três Virtudes Teologais
\end{center}
\begin{itemize}
    \item Fé;
    \item Esperança;
    \item Caridade.
\end{itemize}
\newpage
\begin{center}
    As Quatro Virtudes Cardeais
\end{center}
\begin{itemize}
    \item Prudência;
    \item Justiça;
    \item Fortaleza;
    \item Temperança.
\end{itemize}
\begin{center}
    Os Sete Dons do Espírito Santo
\end{center}
\begin{enumerate}
    \item Sabedoria;
    \item Inteligência;
    \item Conselho;
    \item Fortaleza;
    \item Ciência;
    \item Piedade;
    \item Temor de Deus.
\end{enumerate}
\begin{center}
    Sete Obras de Misericórdia Espirituais
\end{center}
\begin{itemize}
    \item Dar bom conselho;
    \item Ensinar os ignorantes;
    \item Corrigir os que erram;
    \item Consolar os aflitos;
    \item Perdoar as injúrias;
    \item Sofrer com paciência as fraquezas do próximo;
    \item Rogar a Deus pelos vivos e defuntos.
\end{itemize}
\newpage
\begin{center}
    Os Doze Frutos do Espírito Santo
\end{center}
\begin{enumerate}
    \item Caridade;
    \item Paz;
    \item Benignidade;
    \item Longanimidade;
    \item Fidelidade;
    \item Continência;
    \item Alegria;
    \item Paciência;
    \item Bondade;
    \item Mansidão;
    \item Modéstia;
    \item Castidade.
\end{enumerate}
\begin{center}
    Sete Obras de Misericórdia Corporais
\end{center}
\begin{enumerate}
    \item Dar de comer a quem tem fome;
    \item Dar de beber a quem tem sede;
    \item Vestir os nus;
    \item Dar pousada aos peregrinos;
    \item Visitar os enfermos e encarcerados;
    \item Remir os cativos;
    \item Enterrar os mortos.
\end{enumerate}
\newpage
\begin{center}
    As Oito Bem-Aventuranças
\end{center}
\begin{itemize}
    \item Bem-aventurados os pobres em espírito, porque deles é o Reino de Deus.
    \item Bem-aventurados os mansos, porque eles possuirão a terra.
    \item Bem-aventurados os que aflitos, porque serão consolados.
    \item Bem-aventurados os que têm fome e sede de justiça, poque serão saciados.
    \item Bem-aventurados os misericordiosos, porque alcançarão misericórdia.
    \item Bem-aventurados os puros de coração, porque verão a Deus.
    \item Bem-aventurados os que promovem a paz, porque serão chamados filhos de Deus.
    \item Bem-aventurados os que são perseguidos por causa da justiça, porque deles é o Reino dos Céus.
\end{itemize}
\begin{center}
    Os Sete Pecados Capitais e as Virtudes opostas
\end{center}
\begin{enumerate}
    \item Orgulho - Humildade
    \item Avareza - Generosidade
    \item Inveja - Amor ao próximo
    \item Ira - Mansidão
    \item Luxúria - Castidade
    \item Gula - Temperança
    \item Preguiça - Diligência
\end{enumerate}
\begin{center}
    Seis pecados contra o Espírito Santo
\end{center}
\begin{enumerate}
    \item Desesperar da salvação;
    \item Presunção de se salvar sem merecimento;
    \item Contradizer a verdade conhecida por tal;
    \item Ter inveja das mercês que Deus faz a outros;
    \item Obstinação no pecado;
    \item Impenitência final.
\end{enumerate}
\newpage
\begin{center}
    Quatro pecados que bradam ao Céu
\end{center}
\begin{itemize}
    \item Homicídio voluntário;
    \item Pecado sensual contra a natureza;
    \item Opressão dos pobres;
    \item Não pagar a quem trabalha.
\end{itemize}
\begin{center}
    Cooperação e cumplicidade com os pecados alheios
\end{center}
\begin{itemize}
    \item Participando neles direta ou voluntariamente;
    \item Mandando, aconselhando, louvando ou aprovando esses pecados;
    \item Não os revelando ou não os impedindo, quando a isso somos obrigados;
    \item Protegendo os que fazem o mal.
\end{itemize}
\begin{center}
    Os três principais gêneros de boas obras
\end{center}
\begin{itemize}
    \item Oração
    \item Jejum
    \item Esmola
\end{itemize}
\begin{center}
    Conselhos Evangélicos
\end{center}
\begin{itemize}
    \item Pobreza voluntária
    \item Castidade
    \item Obediência
\end{itemize}
\begin{center}
    Os novíssimos
\end{center}
\begin{itemize}
    \item Morte;
    \item Juízo;
    \item Inferno;
    \item Paraíso.
\end{itemize}
\begin{center}
    Dogmas Marianos
\end{center}
\begin{enumerate}
    \item Maternidade divina;
    \item Virgindade perpétua;
    \item Imaculada Conceição;
    \item Assunção da Virgem Maria.
\end{enumerate}
\end{document}
